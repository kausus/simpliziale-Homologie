% geometrische simplizialkomplexe

\section{Geometrische Simplizalkomplexe}

%TODO: bessere zusammenfassung schreiben
Die geometrischen Simpliziale allein reichen nicht aus um beliebige 
Teilmenge des $\R^N$ zu beschreiben bzw. modellieren, es ist noch notwendig eine Ansammlung
von Simpliziale auf richtige art und Weise beschreiben zu können. Hierzu
sind die Simplizialkomplexe da.

\begin{Def}[Geometrischer Simplizialkomplex]
	Wir nennen eine Menge $\Delta$ von (geometrischen) Simplizialen einen \textbf{geometrischen Simplizialkomplex}
	falls er folgende Bedingungen erfüllt:
	\begin{enumerate}[(K1)]
		\item $\emptyset \in \Delta$
		\item Für jeden Simplex $\sigma \in \Delta$ ist auch jede Seite
		$\tau$ in $\Delta$ enthalten
		\item Mit allen Simplexen $\sigma, \sigma' \in \Delta$ ist auch ihr
		Schnitt $\sigma \cap \sigma'$ in $\Delta$ enhalten.
	\end{enumerate}
\end{Def}

\begin{Bsp}
	\item Sei $\sigma^n \coloneqq \{ e_0 \ldots e_n \}$ der Standardsimplex.
		Dieser bildet mit all seinen Seiten einen geometrischen Simplizialkomplex, schreibe hierfür $\Delta^n$.
%TODO: tikz bilder einfügen
\end{Bsp}

%TODO: polytope oder zugrunde liegender raum?
%\begin{Def}[Geometrische Realisierung]
%	Die Geometrische Realisierung eines Simplizalkomplexes $\Delta$ ist
%	die Vereinigung all seiner Elemente.
%	\begin{gather*}
%	|\Delta| \coloneqq \bigcup \left\{ \sigma \; \Big| \; \sigma \in
%	\Delta \right\} \subset \R^{\dim(\Delta)}
%	\end{gather*}
%\end{Def}

\begin{Def}[Unterkomplex, $k$-Skelett, Dimension, Polytop]
  Sei $\Delta$ ein Simplizialkomplex, so definiere
  \begin{enumerate}[1)]
  \item Eine Teilmenge $\Delta'$ von $\Delta$ heißt
    \textbf{Unterkomplex}, falls $\Delta'$ wiederum einen
    Simplizialkomplex bildet.
  \item Die Dimension eines Simplizalkomplexes ist wie folgt
    definiert:
    \begin{gather*}
      \dim(\Delta) \coloneqq \sup \left\{ \dim(\sigma) \; \Big| \;
        \sigma \in \Delta \right\}
    \end{gather*}
    Falls die Dimension unbeschränkt ist, wird diese auf unendlich
    gesetzt.
  \item Das \textbf{$k$-Skelett} $\Delta^{(k)}$ ist der Unterkomplex
    aller Simplizes mit der Dimension kleiner gleich $k$. Der
    Spezialfall $\Delta^{(0)}$ bezeichnet die Eckmenge von $\Delta$.
  \item Das \textbf{Polytop} ist wie folgt definiert
    \begin{gather*}
      |\Delta| \coloneqq \bigcup \left\{ \sigma \; \Big| \; \sigma \in
        \Delta \right\} \subset \R^{\dim(\Delta)}.
    \end{gather*}
    Dies entspricht dem Komplex zugrundeliegenden Raum.
	\end{enumerate}
\end{Def}

Definiere auf dem zugrunde liegenden Raum eines Simplizialkomplexes eine 
Topologie.

%TODO: zeige das dies topologie definiert
\begin{Def}[schwache Topologie]
	Sei $\Delta$ ein Simplizalkomplex, dann definiere die
	\textbf{schwache Topologie} auf der geometrischen Realisierung $| \Delta |$,
	durch folgende Charakterisierung der abgeschlossenen Mengen.
	\begin{gather*}
		A \subset \Delta \text{ abgeschlossen } :\Leftrightarrow A \cap
		\sigma \text{ abgeschlossen für alle } \sigma \in \Delta
	\end{gather*}
	Wobei auf der rechten Seite die Menge abgeschlossen bezüglich der
	Standardtopologie auf dem $\R^{\dim(\Delta)}$ ist.
\end{Def}

\begin{Lem}
  Die schwache Topologie ist feiner als die Standardtopologie und
  falls $\Delta$ endlich ist stimmen Standardtopologie, schwache
  Topologie auf $\gr{\Delta}$ überein.
  \begin{proof}
    Sei \OE\; $\gr{\Delta} \subset \R^N$ und $A \subset \gr{\Delta}$
    eine abgeschlossene Menge, so existiert ein $B \subset \R^N$
    abgeschlossen so dass $A = B \cap \gr{\Delta}$ gilt. Nun gilt für
    jeden Simplex $\sigma \in \Delta$ folgende Gleichheit:
    $A \cap \sigma = B \cap \gr{\Delta} \cap \sigma = B \cap \sigma$
    und somit ist $A \cap \sigma$ abgeschlossen bezüglich $\sigma$,
    also abgeschlossen bezüglich der schwachen Topologie.
		
    Falls nun $\Delta$ eine endliche Menge von Simplizes ist, gilt für
    eine abgeschlossene Menge $A$ bezüglich $\gr{\Delta}$ das
    $A \cap \sigma$ abgeschlossen in $\sigma$ ist und durch
    Vereinigung über die endlichen vielen Simplizes die
    Abgeschlossenheit bezüglich der Standardtopologie erreicht wird.
  \end{proof}
\end{Lem}

\begin{Bem}
  Die schwache Topologie ist im allgemeinen feiner als die
  Spurtopologie auf $\gr{\Delta}$ bezüglich dem $\R^N$. Betrachte
  hierzu den folgenden Simplizialkomplex,
  $\Delta \coloneqq \set{ \{t\} }{t \in \R}$.  Hierbei ist die
  Geometrische Realisierung gleich zu $\R$, aber die schwache
  Topologie entspricht der Diskreten, denn für eine beliebige
  Teilmenge von $A \subset \R$ ist der Schnitt $A \cap \{ t \}$ für
  ein beliebigen $0$-Simplex aus $\Delta$ gleich dem Simplex selbst,
  somit stets abgeschlossen, also ist jede Teilmenge abgeschlossen und
  damit die schwache Topologie auf der Realisierung der diskreten und
  damit ungleich der Standardtopologie auf $\R$.
\end{Bem}

Beweise nun einige Aussagen über die schwache Topologie.

\begin{Satz}
  Sei $\Delta$ ein Simplizialkomplex, so gelten folgende Aussagen
  \begin{enumerate}[(1)]
        \item Die schwache Topologie ist hausdorffsch.
        \item Sei $A \subset \gr{\Delta}$ eine kompakte Teilmenge,
          dann existiert eine endlicher Unterkomplex $\Delta'$ der $A$
          enthält.
	\end{enumerate}
	\begin{proof}
          % TODO: noch zu beweisen
	\end{proof}
\end{Satz}



%definiere (lineare) simpliziale abbildung, isomorphismus von komplexen
%und beweise das jeder endliche simplizialkomplex isomorphm zu einem unterkomplex
% 
% definire \Delta^n für n beliebige kardinalszahl, gilt die obige
% aussage immer noch wenn man unendliche komplexe betrachtet?


%TODO: ziegler - literatur : seite 11 (in reader abgelesen)



% \begin{Def}[Topologie auf dem Träger eines Simplizialkomplexes]
%   Eine Menge $U \subset \left| \Delta \right|$ ist genau dann offen,
%   falls für jeden Simplex $\sigma \in \Delta$ die Menge
%   $U \cap \sigma$ offen im unterliegenden Raum $\R^n$ ist.
% %	also bezüglich der spurtopologie auf \sigma
% % sie elementsOfalgtop seite 111 , review of quotient spaces, für andere beschreibung der topologie
% \end{Def}


%definitionen: unterkomplex, geometrische realisierung, k-skelett, schwache topologie, simpliziale abbildung, simplizialer iso, 

%aussagen: 	-	für endliche komplexe stimmen topologien überein
%			-	|\Delta| ist hausdorffsch
%			- 	Aussagen über kompaktheit, und kompakte teilmengen von \Delta
%			-	












%%% Local Variables:
%%% mode: latex
%%% TeX-master: "main"
%%% End: