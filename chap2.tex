% geometrische simplizialkomplexe

\section{geometrische Simplizalkomplexe}

\begin{Def}[Geometrischer Simplizialkomplex]
	Wir nennen eine Menge $\Delta$ von (geometrischen) Simplizialen einen \textit{geometrischen Simplizialkomplex}
	falls er folgende Bedingungen erfüllt:
	\begin{enumerate}[(K1)]
		\item $\emptyset \in \Delta$
		\item Für jeden Simplex $\sigma \in \Delta$ ist auch jede Seite
		$\tau$ in $\Delta$ enthalten
		\item Mit allen Simplexen $\sigma, \sigma' \in \Delta$ ist auch ihr
		Schnitt in $\Delta$ enhalten.
	\end{enumerate}
\end{Def}


\begin{Def}[Geometrische Realisierung]
	Die Geometrische Realisierung eines Simplizalkomplexes $\Delta$ ist
	die Vereinigung all seiner Elemente, also
	\begin{gather*}
	|\Delta| \coloneqq \bigcup \left\{ \sigma \; \Big| \; \sigma \in
	\Delta \right\} \subset \R^{\dim(\Delta)}
	\end{gather*}
\end{Def}








%%% Local Variables:
%%% mode: latex
%%% TeX-master: "main"
%%% End: