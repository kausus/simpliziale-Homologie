%% abstrakte simplizialkomplexe

\section{Abstrakte Simplizialkomplexe}

Für topologische Anwendung ist der $\R^N$ in vielen Fällen zu konkret und 
unnötig. Deshalb lassen die die Simplizialkomplexe unabhängig vom 
euklidischen Raum definieren, diese abstrakten Simplizialkomplexe 
werden in diesem Abschnitt behandelt.

\begin{Def}
	Eine beliebige, nichtleere Menge $\Delta$ heißt \textbf{abstrakter Simplizialkomplex}
	falls mit jedem Element aus $\Delta$ auch jedes Teilmenge aus diesem Element in $\Delta$
	enthalten ist. Also folgende Bedingung gilt:
	\begin{gather*}
		\sigma \in \Delta \text{ und } \tau \subset \sigma \Rightarrow \tau \in \Delta
	\end{gather*}
\end{Def}

\begin{Bsp}
	\begin{enumerate}[\textbullet]
		\item Für einen geometrischer Simplizial $\sigma$ ist die Menge
			all seiner Seiten ein abstrakter Simplizalkomplex
		\item Die Menge $\Delta = \sset{%
			 \sset{a,b},\sset{a,c},%
			 \sset{b,c},\sset{a},\sset{b},\sset{c},\emptyset}$ ist ein Beispiel
		\item Für eine beliebige Menge ist die Potenzmenge stets ein abstrakter
			Simplizialkomplex
		\item Ein ungerichteter Graph $G=(V,K)$ mit $V$ die Eckmenge und $K$ die Kanten ist ein abstrakter Simplizialkomplex
	\end{enumerate}
\end{Bsp}

% TODO: zeige das für ein element aus \bigcup K ein eindeutiges
% Element aus K existiert so dass das element aus \Int dieses element
% ist




% TODO: eckmenge, k-skelett dimension, n-simplex(kurznotation),
% unterkomplex,

%einfache beispiele







% abzählbare mengen können nur durch 0-simplexe trianguliert werden



%Begriffe:
%geometrisch unabhängig, und linear unabhängig vergleichen gegenüberstellen
% simplex, geometrischer simplizialkomplex, abstrakter simplizialkomplex 


% n-ebene, menge von punkten, aufgespannt durch geo.unab system mit konvexkombinationen
% die faktoren sind eindeutig bestimmt

%affine transformation: x -> Ax+b

%geo.unab systeme werden durch affine transformationen auf geo.unab systeme abgebildet

% n simplex

% baryzentrische koordinaten

% ist die leere menge eine simplizial?

% ein simplex ist genau die konvexe hülle von einer endlichen teilmenge vom \R^N

% eine seite eines simplex ist ein teilsimplex der dimension d

% anzahl der d teilsimplexe ist binomial_koeff(n+1,d+1), für einen n dim simplex
% klar wähle aus den n+1 punkten d+1 stück aus

% verwende das schläfli symbol zur beschreibung der umgebung eines
% punktes aus einem simplex für triangulierungen werden nur 2fache
% schläflisymbole benötigt, also {p,q}, da bei der triangulierung nur
% \laplace^2 simplexe verwendet werden, vereinfacht sich das symbol
% auf die form {3,q}. gebe nun zu jedem punkt auf der
% mannigfaltigkeit, also den 0 dimensionalen simplexen, die anzahl der
% angrenzenden dreiecker/laplace^2 simplexe an


% schreibe in die appendix ein eigenen anhang nur mit getikzten beispielen
% die komplette pflasterung von \R^2 oder allgemein die füllung des \R^n mit \laplace^n
% verschiedene triangulierungen

% zeige nicht-kompakte teilmengen vom \R^n sind nicht triangulierbar,
% einfachstes bsp (0,1), einfach das geometrische realisierungen stets
% kompakte mengen sind, somit kann es keinen homöomorphismus geben


%%% Local Variables:
%%% mode: latex
%%% TeX-master: "main"
%%% End: