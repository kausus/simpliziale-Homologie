%% encodings                                        
\usepackage[T1]{fontenc}                            
\usepackage[utf8]{inputenc}                         
\usepackage[shorthands=off]{babel}                  
\usepackage{lmodern}                                
\usepackage{microtype}                              
\usepackage{scrlayer-scrpage}                       
\usepackage{enumerate}                              
%\usepackage{enumitem}
\usepackage{csquotes}                               
%\usepackage[backend=biber,style=numeric-comp]{biblatex}
%\bibliography{bib.bib}

% for fancy colors
\usepackage{xcolor}

% %maths                                            
\usepackage{amsmath, amsthm, amssymb, bm, bbm, mathtools, dsfont, mathrsfs}

%zum spielen
\usepackage{verbatim}

%commutative diagramms
\usepackage{tikz}

% links, TODO
%\usepackage[pdftitle={Seminar: "K1 von Ringen"},pdfauthor={Julian Seipel}, pdfsubject={Seminararbeit}]{hyperref}
%TODO: abschluss-projekt von tex-kurs heranziehen

%\renewenvironment{proof}{{\bfseries Proof}}{*something*}
% \newenvironment{myproof}[1][\proofname]{%
%   \proof[\ttfamily \scshape \large #1 (yes, ``#1'')]%
% }{\endproof}

%must be loaded as lastest
\usepackage{cleveref}
% out of lemma 1.1 to Lemma 1.1
\let\cref=\Cref                                                    
%special mathsymbols                                

\newcommand{\C}{\mathds{C}}                         
\newcommand{\R}{\mathds{R}}                         
\newcommand{\N}{\mathds{N}}                         
\newcommand{\Z}{\mathds{Z}}                         
\newcommand{\Q}{\mathds{Q}}

% oder alternativ, funktioniert aber nicht
%\defmathbbsymbols{Z Q C}
%\defmathbbsymbolsubs{N R}


\newcommand{\pma}[1]{\begin{pmatrix} #1 \end{pmatrix}}
\newcommand{\A}{\mathscr{A}}
\newcommand{\F}{\mathbb{F}}                                            
\newcommand{\gs}{$\{a_0\ldots a_n\}$ }
\newcommand{\gr}[1]{\left| #1 \right|}
% TODO: schreibe newcommand, das #2 als optional nimmt, und somit kein
% \Big| setzt
\newcommand{\set}[2]{\left\{ #1 \; \Big| \; #2 \right\}}
\newcommand{\sset}[1]{\left\{ #1 \right\}}
\renewcommand{\sp}[2]{\sum\limits_{i=#1}^{#2}}

\newcommand{\nn}[1][\,\bullet\,]{\left\| #1 \right\|}	% norm       
\newcommand{\foo}[2][]{#1 und #2 }

\newcommand{\ch}{\conv(\sset{a_0,\ldots , a_n})}

%\newcommand{\SS}{\mathbb{S}}
                    
%%%%%%%%%%%%%%%%%%%%                                
%theoremstyle                                       
\newtheoremstyle{mythms}
 {15pt}% space above
 {12pt}% space below 
 {\mdseries}% body font
 {}% indent amount
 {\bfseries}% theorem head font
 {:}% punctuation after theorem head
 {0.6cm plus 0.25cm minus 0.1cm}% space after theorem head (\newline possible)
 {}% theorem head spec 


                          
% math (theorems)                     
%\theoremstyle{Definition}             
%\newtheorem{Def}{Definition}[section] 
\theoremstyle{mythms}             
\newtheorem{Def}{Definition}[section] 
\newtheorem{Bsp}[Def]{Beispiel}       

%\theoremstyle{remark}               
\newtheorem{Bem}[Def]{Bemerkung}      
\newtheorem{Wdh}[Def]{Wiederholung}   

%\theoremstyle{plain}                  
\newtheorem{Satz}[Def]{Satz}          
\newtheorem*{satz}{Satz}              
\newtheorem*{Beh}{Behauptung}
\newtheorem{Lem}[Def]{Lemma}          
\newtheorem{Kor}[Def]{Korollar}       
\newtheorem{Fol}[Def]{Folgerung}      

%own color for proof environment
\definecolor{mygray}{rgb}{0.2,0.2,0.2}

%\renewenvironment{center}{\begin{quote}\bfseries}{\end{quote}}
% use for font, [slshape,mdseries]
\renewenvironment{proof}{{ \vspace{0.5cm} \bfseries ~\newline Beweis:\newline}\slshape\color{mygray}}{\hfill$\blacksquare$}



%TODO: schrift soll geändert werden,
%\renewenvironment*{proof}{\bfseries}{}

%stuff used only a few times                        
\DeclareMathOperator{\f}{\varphi} 
\DeclareMathOperator{\GL}{GL}
\DeclareMathOperator{\SL}{SL}
\DeclareMathOperator{\K}{K_1}
\DeclareMathOperator{\SK}{SK_1}
\DeclareMathOperator{\E}{E}
\DeclareMathOperator{\Mat}{Mat}
\DeclareMathOperator{\diag}{diag}
\DeclareMathOperator{\Int}{Int}
\DeclareMathOperator{\B}{B}
\DeclareMathOperator{\Bn}{\overline{\B_{n}(0)}}
\DeclareMathOperator{\Sp}{S} 
\DeclareMathOperator{\V}{V}
\DeclareMathOperator{\conv}{conv}

\DeclarePairedDelimiter{\norm}{\lVert}{\rVert}

%\newcommand{\LLeftrightarrow}{~\Leftrightarrow ~}
%\let\Leftrightarrow=\LLeftrightarrow


%\DeclareMathOperator{\E}{E}
% \newcommand{\exxp}[1]{\exp \left( #1 \right)}       
% \renewcommand{\sin}[1]{\sin \left( #1 \right)}      

%stuff, sometimes needed


%%% Local Variables:
%%% mode: latex
%%% TeX-master: "main"
%%% End:


