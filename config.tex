%% encodings                                        
\usepackage[T1]{fontenc}                            
\usepackage[utf8]{inputenc}                         
\usepackage[shorthands=off]{babel}                  
\usepackage{lmodern}                                
\usepackage{microtype}                              
\usepackage{scrlayer-scrpage}                       
\usepackage{enumerate}                              
%\usepackage{enumitem}
\usepackage{csquotes}                               
%\usepackage[backend=biber,style=numeric-comp]{biblatex}
%\bibliography{bib.bib}

% %maths                                            
\usepackage{amsmath, amsthm, amssymb, bm, bbm, mathtools, dsfont, mathrsfs}

%zum spielen
\usepackage{verbatim}

%commutative diagramms
%\usepackage{tikz-cd}

% links, TODO
%\usepackage[pdftitle={Seminar: "K1 von Ringen"},pdfauthor={Julian Seipel}, pdfsubject={Seminararbeit}]{hyperref}
%TODO: abschluss-projekt von tex-kurs heranziehen

%\renewenvironment{proof}{{\bfseries Proof}}{*something*}
% \newenvironment{myproof}[1][\proofname]{%
%   \proof[\ttfamily \scshape \large #1 (yes, ``#1'')]%
% }{\endproof}

                                                    
%special mathsymbols                                
\newcommand{\C}{\mathds{C}}                         
\newcommand{\R}{\mathds{R}}                         
\newcommand{\N}{\mathds{N}}                         
\newcommand{\Z}{\mathds{Z}}                         
\newcommand{\Q}{\mathds{Q}}
\newcommand{\pma}[1]{\begin{pmatrix} #1 \end{pmatrix}}
\newcommand{\A}{\mathscr{A}}
\newcommand{\F}{\mathbb{F}}                                            
\newcommand{\gs}{$\{a_0\ldots a_n\}$ }
\newcommand{\gr}[1]{\left| #1 \right|}
% TODO: schreibe newcommand, das #2 als optional nimmt, und somit kein
% \Big| setzt
\newcommand{\set}[2]{\left\{ #1 \; \Big| \; #2 \right\}}
%\newcommand{\dim}{Dim}            
%\newcommand{\sup}{sup}
       


%\newcommand{\SS}{\mathbb{S}}
                    
%%%%%%%%%%%%%%%%%%%%                                
%theoremstyle                                       
%\theoremstyle{definition}                           
%\newtheorem*{Def}{Defintion}                        
%\theoremstyle{plain}                               
%\newtheorem{Beh}{Beh}                              
%\newtheorem{Vor}{Vor}                                                    
                          
% math (theorems)                     
\theoremstyle{definition}             
\newtheorem{Def}{Definition}[section] 
\newtheorem{Bsp}[Def]{Beispiel}       

\theoremstyle{remark}               
\newtheorem{Bem}[Def]{Bemerkung}      
\newtheorem{Wdh}[Def]{Wiederholung}   

\theoremstyle{plain}                  
\newtheorem{Satz}[Def]{Satz}          
\newtheorem*{satz}{Satz}              
\newtheorem*{Beh}{Behauptung}
\newtheorem{Lem}[Def]{Lemma}          
\newtheorem{Kor}[Def]{Korollar}       
\newtheorem{Fol}[Def]{Folgerung}      

%TODO: schrift soll geändert werden,
%\renewenvironment*{proof}{\bfseries}{}

%stuff used only a few times                        
\DeclareMathOperator{\f}{\varphi} 
\DeclareMathOperator{\GL}{GL}
\DeclareMathOperator{\SL}{SL}
\DeclareMathOperator{\K}{K_1}
\DeclareMathOperator{\SK}{SK_1}
\DeclareMathOperator{\E}{E}
\DeclareMathOperator{\Mat}{Mat}
\DeclareMathOperator{\diag}{diag}
\DeclareMathOperator{\Int}{Int}
\DeclareMathOperator{\B}{B}
\DeclareMathOperator{\Bn}{\overline{\B_n(0)}}
\DeclareMathOperator{\Sp}{S} 
% \newcommand{\exxp}[1]{\exp \left( #1 \right)}       
% \renewcommand{\sin}[1]{\sin \left( #1 \right)}      

%stuff, sometimes needed


%%% Local Variables:
%%% mode: latex
%%% TeX-master: "main"
%%% End:
