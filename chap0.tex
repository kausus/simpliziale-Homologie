%TODO: Schreibe die konventionen auf die in dieser ausarbeitung verwendet werden, zb. symbole für \R etc, was ist konvex , bla 

%kapitel über konventionen
\section*{Konventionen}

In diesem Seminarvortrag werden folgende Konventionen verwendet.
$\R$ steht für die reellen Zahlen. Eine Menge $A \subset \R^n$ ist konvex
wenn die Verbindungslinie je zweier Punkte vollkommen in $A$ enthalten ist.

Die konvexe Hülle einer Teilmenge eines Vektorraums, ist der Schnitt
aller konvexen Mengen die diese Mengen enthalten. Schreibe hierfür $\conv(A)$.

Es bezeichnet $e_i$ den Einheitsvektor in $\R^n$, wobei $e_0$ als der Nullvektor gesetzt wird.

Eine Konvexkombinationen zweier Vektoren $x,y$ ist eine Linearekombination $\lambda x + \mu y$ mit $\lambda + \mu = 1$.





%%% Local Variables:
%%% mode: latex
%%% TeX-master: "main"
%%% End: