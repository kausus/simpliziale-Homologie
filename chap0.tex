%TODO: Schreibe die konventionen auf die in dieser ausarbeitung verwendet werden, zb. symbole für \R etc, was ist konvex , bla 

%kapitel über konventionen
\section*{Konventionen}

In diesem Seminarvortrag werden folgende Konventionen verwendet.  $0$
ist eine natürliche Zahl.  $\R$ steht für die reellen Zahlen. Eine
Menge $A \subset \R^n$ ist konvex, wenn die Verbindungslinie je zweier
Punkte vollkommen in $A$ enthalten ist.

Die konvexe Hülle einer Teilmenge eines Vektorraums, ist der Schnitt
aller konvexen Mengen, die diese Mengen enthalten. Diese ist
konvex. Schreibe hierfür $\conv(A)$.

Es bezeichnet $e_i$ den Einheitsvektor in $\R^n$, wobei $e_0$ als der Nullvektor gesetzt wird.

Die Konvexkombination zweier Vektoren $x$ und $y$ ist eine
Linearekombination $\lambda x + \mu y$ mit $\lambda + \mu = 1$.

Die Potenzmenge wird durch $\pow$ geschrieben.

Für eine Menge $A \subset X$ aus einem topologischen Raum, bezeichnet
$\overline{A}$ den Abschluss von $A$.

Das Zeichen \OE~ bedeutet ohne Einschränkung.

Die $1$-Norm $\nn_1$ auf dem $\R^N$ ist wie folgt definiert:
$\nn[x]_1 \coloneqq \sp{1}{N} |x_i|$.



%%% Local Variables:
%%% mode: latex
%%% TeX-master: "main"
%%% End: