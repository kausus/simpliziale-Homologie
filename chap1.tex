% themenbereiche
%Geometrische simpliziale Komplexe; Triangulierungen; abstrakte
%simpliziale Komplexe; Beispiele simplizialer Komplexe

%NOTIZEN:
% warum simpliziale komplexe,mengen
% selbe homotophietheorie wie der ursprüngliche top raum

%warum simpliziale komplexe bzw mengen gebraucht werden, selbe homotophietheorie und selbe homotopiegruppen, einfachere berechnung der homotopiegruppen.

%homotopiegruppen sind invarianten von top räumen, einfaches kriterium
%zur unterscheidung von top räumen

% 

\section{Geometrische Simplizialkomplexe}


Wir definieren zunächst grundlegende Begriffe zu den Simplizialen.

\begin{Def}[Geometrisch unabhängig\footnote{Oder auch affine Unabhängigkeit}]
	Eine Menge $\{ a_0\ldots a_n \} \subset \R^N$ heißt 
	geometrisch unabhängig, falls das System von Vektoren
	\begin{gather*}
		a_0 - a_1 ,  \ldots , a_0 - a_n
	\end{gather*}
	unabhängig im Sinne der Linearen Algebra ist.
\end{Def}

%\begin{Bem}
%	Man nennt die geometrische Unabhängigkeit auch affine Unabhängigkeit
%\end{Bem}

\begin{Def}[geometrischer $n$-Simplex]
	Zu einem geometrisch unabhängigen System \gs, nennt man die Menge
	\begin{gather*}
		(a_0 \ldots a_n) = 
		\left\{ 
			 \sum\limits_{i=0}^n t_i a_i \; \Big| \; % finde bessere
			 t_i \in [0,1] \; , \;  \sum\limits_{i=0}^n t_i = 1
		\right\}
	\end{gather*}
	den (geometrischen) $n$-Simplex und schreibt $\sigma^n$.
\end{Def}

%TODO: erstes Beispiel zu einem simplex, zeiche in tikz \sigma^k für k =0,1,2,3

Nach diesen ersten einfachen Beispielen definiere weitere wichtige 
Begriffsbildungen.

\begin{Def}[Baryzentrische Koordinaten, Seite]
	Es bezeichnet $x$ einen Punkt aus einem geometrischen Simplex $\sigma$. In der Darstellung $x = \sum\limits_{i=0}^n t_i a_i$ 
	nennt man die $t_i$, die \textbf{baryzentrischen} Koordinaten.
	
	Eine Teilmenge eines Simplex, heißt \textbf{Seite} falls diese wiederum einen Simplex bildet. Eine Seite heißt echt, falls sie verschieden von $\sigma$ ist.
\end{Def}

\begin{Def}[Eckmenge, Rand, Inneres]
	Sei $\sigma$ ein geometrischer $n$-Simplex, so definiere folgende
	geometrische Objekte
%TODO: soll die nummerierung fett gehalten sein
	\begin{enumerate}[{\bfseries1)}] 
		\item Die Menge $\{  a_0 , \ldots , a_n \}$ bezeichnet man als Eckmenge $V(\sigma)$
		\item Der Rand von $\sigma$ ist die folgende Menge:
			\begin{gather*}
				\partial\sigma = \bigcup \; \left\{ \tau  \; \Big| \; \tau \text{ ist  echte Seite von } \sigma \right\}
			\end{gather*} 
		\item Das Innere Des Simplex ist $\Int(\sigma) \coloneqq 
		\sigma \setminus \partial\sigma$
	\end{enumerate}
\end{Def}

Es werden nun einige Charakterisierungen und Aussagen über die Simplizes und deren geometrischen Objekten bewiesen.

\begin{Satz}
	\upshape Sei $\sigma = (a_0 , \ldots , a_n)$ ein $n$-Simplex, $x \in \sigma$ mit der baryzentrischen Darstellung $x=\sum\limits_{i=0}^n t_i a_i$ , so gelten folgende Aussagen:
	\begin{enumerate}[1)]%[{\scshape1)}]
		\item $x \in \partial\sigma \Leftrightarrow \exists \; 0 \leq i \leq n : t_i = 0$
		\item $x \in \Int(\sigma) \Leftrightarrow \forall \; 0 \leq i \leq n : t_i > 0$
%TODO: entweder \in (0,1) oder t_i \geq 0
		\item  Die baryzentrischen Koordinaten $t_i(x)$ als Funktionen, sind stetig in $x$
		\item $\sigma$ ist eine konvexe, kompakte Teilmenge vom $\R^N$ und $\Int(\sigma)$ ist offen und konvex.
		\item Es gilt $\overline{\Int(\sigma)} = \sigma$
		\item Es gibt einen Homöomorphismus $\sigma \simeq \Bn$ der 
		das Innere $\Int{\sigma}$ auf die $\Sp^{n-1}$ abbildet
	\end{enumerate}
	\begin{proof}
		foo
%TODO: noch zu beweisen
	\end{proof}
\end{Satz}



%TODO: eckmenge, k-skelett	dimension, n-simplex(kurznotation), unterkomplex, 
\begin{Def}[Geometrischer Simplizialkomplex]
  Wir nennen $\Delta$ einen \textit{geometrischen Simplizialkomplex} 
  falls er folgende Bedingungen erfüllt:
  \begin{enumerate}[(K1)]
  	\item $\emptyset \in \Delta$
  	\item Für jeden Simplex $\sigma \in \Delta$ ist auch jede Seite $\tau$ in $\Delta$ enthalten
  	\item Mit allen Simplexen $\sigma, \sigma' \in \Delta$ ist auch ihr Schnitt in $\Delta$ enhalten.
  \end{enumerate}
\end{Def}
%einfache beispiele

\begin{Def}[Topologie auf dem Träger eines Simplizialkomplexes]
	Eine Menge $U \subset  \left| \Delta \right|$ ist genau dann offen, falls für jeden Simplex $\sigma \in \Delta$ die Menge $U \cap \sigma$ offen im unterliegenden Raum $\R^n$ ist. 
%	also bezüglich der spurtopologie auf \sigma
% sie elementsOfalgtop seite 111 , review of quotient spaces, für andere beschreibung der topologie
\end{Def}

%TODO: topologie von sim komplexen ist hausdorffsch

%TODO: ziegler - literatur : seite 11 (in reader abgelesen)


% abzählbare mengen können nur durch 0-simplexe trianguliert werden



%Begriffe:
%geometrisch unabhängig, und linear unabhängig vergleichen gegenüberstellen
% simplex, geometrischer simplizialkomplex, abstrakter simplizialkomplex 


% n-ebene, menge von punkten, aufgespannt durch geo.unab system mit konvexkombinationen
% die faktoren sind eindeutig bestimmt

%affine transformation: x -> Ax+b

%geo.unab systeme werden durch affine transformationen auf geo.unab systeme abgebildet

% n simplex

% baryzentrische koordinaten

% ist die leere menge eine simplizial?

% ein simplex ist genau die konvexe hülle von einer endlichen teilmenge vom \R^N

% eine seite eines simplex ist ein teilsimplex der dimension d

% anzahl der d teilsimplexe ist binomial_koeff(n+1,d+1), für einen n dim simplex
% klar wähle aus den n+1 punkten d+1 stück aus

% verwende das schläfli symbol zur beschreibung der umgebung eines
% punktes aus einem simplex für triangulierungen werden nur 2fache
% schläflisymbole benötigt, also {p,q}, da bei der triangulierung nur
% \laplace^2 simplexe verwendet werden, vereinfacht sich das symbol
% auf die form {3,q}. gebe nun zu jedem punkt auf der
% mannigfaltigkeit, also den 0 dimensionalen simplexen, die anzahl der
% angrenzenden dreiecker/laplace^2 simplexe an


% schreibe in die appendix ein eigenen anhang nur mit getikzten beispielen
% die komplette pflasterung von \R^2 oder allgemein die füllung des \R^n mit \laplace^n
% verschiedene triangulierungen

% zeige nicht-kompakte teilmengen vom \R^n sind nicht triangulierbar,
% einfachstes bsp (0,1), einfach das geometrische realisierungen stets
% kompakte mengen sind, somit kann es keinen homöomorphismus geben




%%% Local Variables:
%%% mode: latex
%%% TeX-master: "main"
%%% End:
