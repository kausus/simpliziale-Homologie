% themenbereiche
%Geometrische simpliziale Komplexe; Triangulierungen; abstrakte
%simpliziale Komplexe; Beispiele simplizialer Komplexe

%NOTIZEN:
% warum simpliziale komplexe,mengen
% selbe homotophietheorie wie der ursprüngliche top raum

%warum simpliziale komplexe bzw mengen gebraucht werden, selbe homotophietheorie und selbe homotopiegruppen, einfachere berechnung der homotopiegruppen.

%homotopiegruppen sind invarianten von top räumen, einfaches kriterium
%zur unterscheidung von top räumen

%TODO: Schreibe die konventionen auf die in dieser ausarbeitung verwendet werden, zb. symbole für \R etc, was ist konvex , bla 
\section{Geometrische Simplizialkomplexe}

simpliziale sind rein kombinatorische objekte, mit denen man auf gute
art und weise topologische räume beschreiben kann, insbesondere ihre
homologiegruppen ausrechnen kann.

\subsection{Geometrische Simpliziale}


Wir definieren zunächst grundlegende Begriffe zu den Simplizialen.

\begin{Def}[Geometrisch unabhängig\footnote{Oder auch affine Unabhängigkeit}]
  Eine Menge $\{ a_0\ldots a_n \} \subset \R^N$ heißt geometrisch
  unabhängig, falls das System von Vektoren
  \begin{gather*}
    a_0 - a_1 , \ldots , a_0 - a_n
  \end{gather*}
  unabhängig im Sinne der Linearen Algebra ist.
\end{Def}

% \begin{Bem}
%	Man nennt die geometrische Unabhängigkeit auch affine Unabhängigkeit
%\end{Bem}

\begin{Def}[geometrischer $n$-Simplex]
  Zu einem geometrisch unabhängigen System \gs, nennt man die Menge
  \begin{gather*}
    (a_0 \ldots a_n) = \left\{ \sum\limits_{i=0}^n t_i a_i \; \Big|
      \; % finde bessere
      t_i \in [0,1] \; , \; \sum\limits_{i=0}^n t_i = 1 \right\}
  \end{gather*}
  den (geometrischen) $n$-Simplex und schreibt $\sigma^n$.
\end{Def}

%TODO: erstes Beispiel zu einem simplex, zeiche in tikz \sigma^k für k =0,1,2,3

Nach diesen ersten einfachen Beispielen definiere weitere wichtige
Begriffsbildungen.

\begin{Def}[Baryzentrische Koordinaten, Seite]
  Es bezeichnet $x$ einen Punkt aus einem geometrischen Simplex
  $\sigma$. In der Darstellung $x = \sum\limits_{i=0}^n t_i a_i$ nennt
  man die $t_i$, die \textbf{baryzentrischen} Koordinaten.
	
  Eine Teilmenge eines Simplex, heißt \textbf{Seite} falls diese
  wiederum einen Simplex bildet. Eine Seite heißt echt, falls sie
  verschieden von $\sigma$ ist.
\end{Def}

\begin{Def}[Eckmenge, Rand, Inneres]
  Sei $\sigma$ ein geometrischer $n$-Simplex, so definiere folgende
  geometrische Objekte
%TODO: soll die nummerierung fett gehalten sein
  \begin{enumerate}[{\bfseries1)}]
  \item Die Menge $\{ a_0 , \ldots , a_n \}$ bezeichnet man als
    Eckmenge $V(\sigma)$
  \item Der Rand von $\sigma$ ist die folgende Menge:
    \begin{gather*}
      \partial\sigma = \bigcup \; \left\{ \tau \; \Big| \; \tau \text{
          ist echte Seite von } \sigma \right\}
    \end{gather*}
  \item Das Innere Des Simplex ist
    $\Int(\sigma) \coloneqq \sigma \setminus \partial\sigma$
  \end{enumerate}
\end{Def}

Es werden nun einige Charakterisierungen und Aussagen über die
Simplizes und deren geometrischen Objekten bewiesen.

\begin{Satz}
  \upshape Sei $\sigma = (a_0 , \ldots , a_n)$ ein $n$-Simplex,
  $x \in \sigma$ mit der baryzentrischen Darstellung
  $x=\sum\limits_{i=0}^n t_i a_i$ , so gelten folgende Aussagen:
  \begin{enumerate}[1)]%[{\scshape1)}]
  \item
    $x \in \partial\sigma \Leftrightarrow \exists \; 0 \leq i \leq n :
    t_i = 0$
  \item
    $x \in \Int(\sigma) \Leftrightarrow \forall \; 0 \leq i \leq n :
    t_i > 0$
    % TODO: entweder \in (0,1) oder t_i \geq 0
  \item Die baryzentrischen Koordinaten $t_i(x)$ als Funktionen, sind
    stetig in $x$ und durch $x$ eindeutig bestimmt
  \item $\sigma$ ist eine konvexe, kompakte Teilmenge vom $\R^N$ und
    $\Int(\sigma)$ ist offen und konvex.
  \item Es gilt $\overline{\Int(\sigma)} = \sigma$
  \item Es gibt einen Homöomorphismus $\sigma \simeq \Bn$ der das
    Innere $\Int{\sigma}$ auf die $\Sp^{n-1}$ abbildet
  \end{enumerate}
  \begin{proof}
    foo
%TODO: noch zu beweisen
  \end{proof}
\end{Satz}


\subsection{Simplizialkomplex}

Definiere nun auf sinnvolle art und Weise eine Ansammlung von
geometrischen Simplizialen, einen (geometrischen) Simplizialkomplex.

\begin{Def}
  Eine Menge $\Delta$ von (geometrischen) Simplizialen heißt
  \textbf{Simplizialkomplex} falls die folgenden Bedingungen erfüllt
  sind.
  \begin{enumerate}[1)]
  \item $\emptyset \in \Delta$
  \item Mit einem Simplex $\sigma$ aus $\Delta$ sind auch alle Seiten
    aus $\Delta$.
  \item Für je zwei Simplizies aus $\Delta$ ist auch ihr Schnitt
    wiederum in $\Delta$.
  \end{enumerate}
\end{Def}

% TODO: zeige das für ein element aus \bigcup K ein eindeutiges
% Element aus K existiert so dass das element aus \Int dieses element
% ist

\begin{Def}[Teilkomplex, $k$-Skelett, Dimension]
  Sei $\Delta$ ein Simplizialkomplex, so definiere
  \begin{enumerate}[1)]
  \item Eine Teilmenge $\Delta'$ von $\Delta$ heißt Teilkomplex, falls
    $\Delta'$ diese Menge wiederum einen Simplizialkomplex bildet.
  \item Die Dimension eines Simplizalkomplexes ist wie folgt
    definiert:
    \begin{gather*}
      \dim(\Delta) \coloneqq \sup \left\{ \dim(\sigma) \; \Big| \;
        \sigma \in \Delta \right\}
    \end{gather*}
    die Dimension kann auch unendlich sein.
  \item Das $k$-Skelett $\Delta^k$ ist der Teilkomplex aller Simplizes
    mit der Dimension kleiner gleich $k$.  Ein Spezialfall davon ist
    die Eckmenge, dies entspr
  \end{enumerate}
\end{Def}


% TODO: eckmenge, k-skelett dimension, n-simplex(kurznotation),
% unterkomplex,
\begin{Def}[Geometrischer Simplizialkomplex]
  Wir nennen $\Delta$ einen \textit{geometrischen Simplizialkomplex}
  falls er folgende Bedingungen erfüllt:
  \begin{enumerate}[(K1)]
  \item $\emptyset \in \Delta$
  \item Für jeden Simplex $\sigma \in \Delta$ ist auch jede Seite
    $\tau$ in $\Delta$ enthalten
  \item Mit allen Simplexen $\sigma, \sigma' \in \Delta$ ist auch ihr
    Schnitt in $\Delta$ enhalten.
  \end{enumerate}
\end{Def}
%einfache beispiele



\begin{Def}[Geometrische Realisierung]
  Die Geometrische Realisierung eines Simplizalkomplexes $\Delta$ ist
  die Vereinigung all seiner Elemente, also
  \begin{gather*}
    |\Delta| \coloneqq \bigcup \left\{ \sigma \; \Big| \; \sigma \in
      \Delta \right\} \subset \R^{\dim(\Delta)}
  \end{gather*}
\end{Def}

%TODO: zeige das dies topologie definiert
\begin{Def}[schwache Topologie]
  Sei $\Delta$ ein Simplizalkomplex, dann definiere die
  \textbf{schwache Topologie} auf der geometrischen Realisierung $| \Delta |$,
  durch folgende Charakterisierung der abgeschlossenen Mengen.
  \begin{gather*}
    A \subset \Delta \text{ abgeschlossen } :\Leftrightarrow A \cap
    \sigma \text{ abgeschlossen für alle } \sigma \in \Delta
  \end{gather*}
  Wobei auf der rechten Seite die Menge abgeschlossen bezüglich der
  Standardtopologie auf dem $\R^{\dim(\Delta)}$ ist.
\end{Def}

\begin{Lem}
  Die schwache Topologie ist feiner als die Standardtopologie und
  falls $\Delta$ endlich ist stimmen Standardtopologie, schwache
  Topologie auf $\gr{\Delta}$ überein.
  \begin{proof}
    Sei \OE\; $\gr{\Delta} \subset \R^N$ und $A \subset \gr{\Delta}$
    eine abgeschlossene Menge, so existiert ein $B \subset \R^N$
    abgeschlossen so dass $A = B \cap \gr{\Delta}$ gilt. Nun gilt für
    jeden Simplex $\sigma \in \Delta$ folgende Gleichheit:
    $A \cap \sigma = B \cap \gr{\Delta} \cap \sigma = B \cap \sigma$
    und somit ist $A \cap \sigma$ abgeschlossen bezüglich $\sigma$,
    also abgeschlossen bezüglich der schwachen Topologie.

    Falls nun $\Delta$ eine endliche Menge von Simplizes ist, gilt für
    eine abgeschlossene Menge $A$ bezüglich $\gr{\Delta}$ das
    $A \cap \sigma$ abgeschlossen in $\sigma$ ist und durch
    Vereinigung über die endlichen vielen Simplizes die
    Abgeschlossenheit bezüglich der Standardtopologie erreicht wird.
  \end{proof}
\end{Lem}

\begin{Bem}
  Die schwache Topologie ist im allgemeinen feiner als die
  Spurtopologie auf $\gr{\Delta}$ bezüglich dem $\R^N$. Betrachte
  hierzu den folgenden Simplizialkomplex,
  $\Delta \coloneqq \set{ \{t\} }{t \in \R}$.  Hierbei ist die
  Geometrische Realisierung gleich zu $\R$, aber die schwache
  Topologie entspricht der Diskreten, denn für eine beliebige
  Teilmenge von $A \subset \R$ ist der Schnitt $A \cap \{ t \}$ für
  ein beliebigen $0$-Simplex aus $\Delta$ gleich dem Simplex selbst,
  somit stets abgeschlossen, also ist jede Teilmenge abgeschlossen und
  damit die schwache Topologie auf der Realisierung der diskreten und
  damit ungleich der Standardtopologie auf $\R$.
\end{Bem}



% \begin{Def}[Topologie auf dem Träger eines Simplizialkomplexes]
%   Eine Menge $U \subset \left| \Delta \right|$ ist genau dann offen,
%   falls für jeden Simplex $\sigma \in \Delta$ die Menge
%   $U \cap \sigma$ offen im unterliegenden Raum $\R^n$ ist.
% %	also bezüglich der spurtopologie auf \sigma
% % sie elementsOfalgtop seite 111 , review of quotient spaces, für andere beschreibung der topologie
% \end{Def}

%TODO: topologie von sim komplexen ist hausdorffsch

%TODO: ziegler - literatur : seite 11 (in reader abgelesen)


% abzählbare mengen können nur durch 0-simplexe trianguliert werden



%Begriffe:
%geometrisch unabhängig, und linear unabhängig vergleichen gegenüberstellen
% simplex, geometrischer simplizialkomplex, abstrakter simplizialkomplex 


% n-ebene, menge von punkten, aufgespannt durch geo.unab system mit konvexkombinationen
% die faktoren sind eindeutig bestimmt

%affine transformation: x -> Ax+b

%geo.unab systeme werden durch affine transformationen auf geo.unab systeme abgebildet

% n simplex

% baryzentrische koordinaten

% ist die leere menge eine simplizial?

% ein simplex ist genau die konvexe hülle von einer endlichen teilmenge vom \R^N

% eine seite eines simplex ist ein teilsimplex der dimension d

% anzahl der d teilsimplexe ist binomial_koeff(n+1,d+1), für einen n dim simplex
% klar wähle aus den n+1 punkten d+1 stück aus

% verwende das schläfli symbol zur beschreibung der umgebung eines
% punktes aus einem simplex für triangulierungen werden nur 2fache
% schläflisymbole benötigt, also {p,q}, da bei der triangulierung nur
% \laplace^2 simplexe verwendet werden, vereinfacht sich das symbol
% auf die form {3,q}. gebe nun zu jedem punkt auf der
% mannigfaltigkeit, also den 0 dimensionalen simplexen, die anzahl der
% angrenzenden dreiecker/laplace^2 simplexe an


% schreibe in die appendix ein eigenen anhang nur mit getikzten beispielen
% die komplette pflasterung von \R^2 oder allgemein die füllung des \R^n mit \laplace^n
% verschiedene triangulierungen

% zeige nicht-kompakte teilmengen vom \R^n sind nicht triangulierbar,
% einfachstes bsp (0,1), einfach das geometrische realisierungen stets
% kompakte mengen sind, somit kann es keinen homöomorphismus geben




%%% Local Variables:
%%% mode: latex
%%% TeX-master: "main"
%%% End:
