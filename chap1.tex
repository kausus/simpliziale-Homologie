% themenbereiche
%Geometrische simpliziale Komplexe; Triangulierungen; abstrakte
%simpliziale Komplexe; Beispiele simplizialer Komplexe


%homotopiegruppen sind invarianten von top räumen, einfaches kriterium
%zur unterscheidung von top räumen


% schlagwort verzeichnis
%TODO: folgende begriffe müssen definiert, verwendet und verstanden sein: simplex, seite, eckmenge, dimension, simplizialkomplex, n-skelett, geometrischer simplizialkomplex, geometrische realisierung, schwache topologie, triangulierbar, simpliziale abbildung, baryzentrische koordinaten, 

%warum simpliziale komplexe bzw mengen gebraucht werden, selbe homotophietheorie und selbe homotopiegruppen, einfachere berechnung der homotopiegruppen.

\section{Geometrische Simpliziale}

Simpliziale sind rein kombinatorische Objekte, mit denen man auf gute
art und Weise topologische Räume beschreiben kann, insbesondere ihre
Homologiegruppen berechnen kann.

Wir definieren zunächst grundlegende Begriffe zu den Simplizialen.

\begin{Def}[Geometrisch unabhängig\footnote{Oder auch affine Unabhängigkeit}]
  Eine Menge $\{ a_0\ldots a_n \} \subset \R^N$ heißt \textbf{geometrisch %
  unabhängig}, falls das System von Vektoren
  \begin{gather*}
    a_0 - a_1 , \ldots , a_0 - a_n
  \end{gather*}
  unabhängig im Sinne der Linearen Algebra ist.
\end{Def}

\begin{Lem}
  Teilsysteme von geometrisch unabhängigen Systemen sind geometrisch
  unabhängig und für ein Menge $\{ a_0 , \ldots , a_n \} \subset \R^N$ sind folgende Aussagen zueinander äquivalent:
  \begin{enumerate}[i)]
  	\item $\{ a_0 , \ldots , a_n \}$ ist geometrische unabhängig
  	\item Für $\sum\limits_{i=0}^n t_i = 0$ und $\sum\limits_{i=0}^n t_i a_i = 0$ folgt stets $t_i = 0$ für alle $i \in \{ 0,\ldots,n\}$
  \end{enumerate}
  \begin{proof}
    Sei $\{ a_0 , \ldots , a_n \}$ ein geometrisch unabhängiges System und hierzu 
    ein Teilsystem $\{ a_{i_0},\ldots,a_{i_r} \}$. 
    Sei \OE~ $i_0 = 0$ , sonst nummeriere um oder betrachte ein anderes $i_j$, so dass für dieses $i_j$ das ursprüngliche geometrisch unabhängige System der Punkt $a_{i_j}$ als Basispunkt gewählt werden kann. Nun ist nach Definition $ a_0 - a_1 , \ldots , a_0 - a_n$ linear unabhängig und somit auch das Teilsystem $ a_0 - a_ {i_1}, \ldots , a_0 - a_{i_r}$.
    Für die Äquivalenz:
%TODO: beweis noch fertig machen
  \end{proof}
\end{Lem}

In der nächsten Definition wird das diesem Vortrag zugrundeliegende Objekt von Interesse definiert.

\begin{Def}[geometrischer $n$-Simplex]
  Zu einem geometrisch unabhängigen System \gs $\subset \R^N$, nennt man die Menge
  \begin{gather*}
    (a_0 \ldots a_n) 
      =
    \set{\sum\limits_{i=0}^n t_i a_i \in \R^N}{ t_i \in [0,1] ~,~ \sum\limits_{i=0}^n t_i = 1}
  \end{gather*}
  den (geometrischen) $n$-Simplex und schreibt $\sigma^n$. Dies ist die Menge aller Konvexkombinationen des Systems \gs. Als Konvention 
  ist $\sigma^n$ stets in den $\R^N$ eingebettet für $n \leq N$.
\end{Def}

%TODO: erstes Beispiel zu einem simplex, zeiche in tikz \sigma^k für k =0,1,2,3

Nach diesen ersten einfachen Beispielen definiere weitere wichtige
Begriffsbildungen.

\begin{Lem}[Baryzentrische Koordinaten]
  \normalfont Es bezeichnet $x$ einen Punkt aus einem geometrischen Simplex
  $\sigma^n$. In der Darstellung $x = \sum\limits_{i=0}^n t_i a_i$ nennt
  man die $t_i$, die \textbf{baryzentrischen} Koordinaten. Diese sind durch $x$
  eindeutig bestimmt und als Funktionen $t_i : \sigma^n \rightarrow [0,1]$ stetig.
  \begin{proof}
  	Beweise zunächst die Eindeutigkeit über eine äquivalente Formulierung der
  	geometrischen Unabhängigkeit 
  	\begin{description}
  		\item[Äquivalenz: ] Es gilt:
	  		\begin{Beh}
	  			bla
	  		\end{Beh}
  		\item[Eindeutigkeit: ] 
  		\item[Stetigkeit: ]
  	\end{description}
  \end{proof}
	
	
%  Eine Teilmenge eines Simplex, heißt \textbf{Seite} falls diese
%  wiederum einen Simplex bildet. Eine Seite heißt echt, falls sie
%  verschieden von $\sigma$ ist.
\end{Lem}

\begin{Def}[Eckmenge, Dimension, Seite, Rand, Inneres]
  Sei $\sigma$ ein geometrischer $n$-Simplex, so definiere folgende
  geometrische Objekte
%TODO: soll die nummerierung fett gehalten sein
  \begin{enumerate}[\textbullet]%{\bfseries1)}]
  \item Die Menge $\{ a_0 , \ldots , a_n \}$ bezeichnet man als
    \textbf{Eckmenge} $\V(\sigma)$ von $\sigma$.
  \item Ein Teilmenge $\tau \subset \sigma$ heißt \textbf{Seite} falls
	  $\tau$ einen Simplex bildet. Eine Seite heißt \textbf{echt} falls sie von 
		$\sigma$ verschieden ist.
	\item Die \textbf{Dimension} von $\sigma$ ist die Zahl $n$ bzw. 
		$\dim(\sigma) = \gr{\V(\sigma)} - 1$
  \item Der \textbf{Rand} von $\sigma$ ist die folgende Menge:
    \begin{gather*}
      \partial\sigma = \bigcup \; \left\{ \tau \; \Big| \; \tau \text{
          ist echte Seite von } \sigma \right\}
    \end{gather*}
  \item Das \textbf{Innere} des Simplex ist die Menge
    \begin{gather*}
    	\Int(\sigma) \coloneqq \sigma \setminus \partial\sigma
    \end{gather*}
  \end{enumerate}
\end{Def}

%TODO: tikz bilder einfügen von beispielen in denen für ein beispiel symbolisch die obrigen definitionen angegeben werden

Es werden nun einige Charakterisierungen und Aussagen über die
Simplizes und deren geometrischen Objekten bewiesen.

\begin{Satz}
  \normalfont Sei $\sigma = \{ a_0 \ldots a_n \}$ ein $n$-Simplex,
  $x \in \sigma$ mit der baryzentrischen Darstellung
  $x=\sum\limits_{i=0}^n t_i a_i$ , so gelten folgende Aussagen:
  \begin{enumerate}[(a)]
  \item
    $x \in \partial\sigma \Leftrightarrow \exists \; 0 \leq i \leq n :
    t_i = 0$
  \item
    $x \in \Int(\sigma) \Leftrightarrow \forall \; 0 \leq i \leq n :
    t_i > 0$
  \item Jeder Simplex $\sigma$ ist eine konvexe, kompakte Teilmenge (
    vom $\R^N$), insbesondere ist die konvexe Hülle von
    $\{ a_0 \ldots a_n \}$ identisch mit dem Simplex $\sigma$
  \item Das Innere $\Int(\sigma)$ ist offen und konvex
  \item Es gilt $\overline{\Int(\sigma)} = \sigma$
  \item Es gibt einen Homöomorphismus $\sigma \simeq \Bn$ der das
    Innere $\Int{\sigma}$ auf die $\Sp^{n-1}$ abbildet
  \end{enumerate}
  \begin{proof}
    foo
%TODO: noch zu beweisen
  \end{proof}
\end{Satz}

%TODO: definiere die simpliziale auch für \R^J für J eine beliebige Menge
%also auch für unendliche dimension






%%% Local Variables:
%%% mode: latex
%%% TeX-master: "main"
%%% End:
