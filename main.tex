\documentclass[paper=a4, twoside=false, twocolumn=false, ngerman, fontsize=10pt, titlepage=false, parskip=half, abstract=false,toc=flat,left,listof,indent]{scrartcl}

%\makeindex

%TODO: vll den Seitenkopf/fuss noch bearbeiten
%\pagestyle{scrheadings}
%\ofoot[{\includegraphics[height=10ex]{unilogo.pdf}}]{\includegraphics[height=10ex]{unilogo.pdf}}

%% encodings                                        
\usepackage[T1]{fontenc}                            
\usepackage[utf8]{inputenc}                         
\usepackage[shorthands=off]{babel}                  
\usepackage{lmodern}                                
\usepackage{microtype}                              
\usepackage{scrlayer-scrpage}                       
\usepackage{enumerate}                              
%\usepackage{enumitem}
\usepackage{csquotes}                               
%\usepackage[backend=biber,style=numeric-comp]{biblatex}
%\bibliography{bib.bib}



% %maths                                            
\usepackage{amsmath, amsthm, amssymb, bm, bbm, mathtools, dsfont, mathrsfs}

%zum spielen
\usepackage{verbatim}

%commutative diagramms
%\usepackage{tikz-cd}

% links, TODO
%\usepackage[pdftitle={Seminar: "K1 von Ringen"},pdfauthor={Julian Seipel}, pdfsubject={Seminararbeit}]{hyperref}
%TODO: abschluss-projekt von tex-kurs heranziehen

%\renewenvironment{proof}{{\bfseries Proof}}{*something*}
% \newenvironment{myproof}[1][\proofname]{%
%   \proof[\ttfamily \scshape \large #1 (yes, ``#1'')]%
% }{\endproof}

%must be loaded as lastest
\usepackage{cleveref}
                                                    
%special mathsymbols                                
\newcommand{\C}{\mathds{C}}                         
\newcommand{\R}{\mathds{R}}                         
\newcommand{\N}{\mathds{N}}                         
\newcommand{\Z}{\mathds{Z}}                         
\newcommand{\Q}{\mathds{Q}}
\newcommand{\pma}[1]{\begin{pmatrix} #1 \end{pmatrix}}
\newcommand{\A}{\mathscr{A}}
\newcommand{\F}{\mathbb{F}}                                            
\newcommand{\gs}{$\{a_0\ldots a_n\}$ }
\newcommand{\gr}[1]{\left| #1 \right|}
% TODO: schreibe newcommand, das #2 als optional nimmt, und somit kein
% \Big| setzt
\newcommand{\set}[2]{\left\{ #1 \; \Big| \; #2 \right\}}
\newcommand{\sset}[1]{\left\{ #1 \right\}}
\renewcommand{\sp}[2]{\sum\limits_{i=#1}^{#2}}
       


%\newcommand{\SS}{\mathbb{S}}
                    
%%%%%%%%%%%%%%%%%%%%                                
%theoremstyle                                       
%\theoremstyle{definition}                           
%\newtheorem*{Def}{Defintion}                        
%\theoremstyle{plain}                               
%\newtheorem{Beh}{Beh}                              
%\newtheorem{Vor}{Vor}                                                    
                          
% math (theorems)                     
\theoremstyle{definition}             
\newtheorem{Def}{Definition}[section] 
\newtheorem{Bsp}[Def]{Beispiel}       

\theoremstyle{remark}               
\newtheorem{Bem}[Def]{Bemerkung}      
\newtheorem{Wdh}[Def]{Wiederholung}   

\theoremstyle{plain}                  
\newtheorem{Satz}[Def]{Satz}          
\newtheorem*{satz}{Satz}              
\newtheorem*{Beh}{Behauptung}
\newtheorem{Lem}[Def]{Lemma}          
\newtheorem{Kor}[Def]{Korollar}       
\newtheorem{Fol}[Def]{Folgerung}      

%TODO: schrift soll geändert werden,
%\renewenvironment*{proof}{\bfseries}{}

%stuff used only a few times                        
\DeclareMathOperator{\f}{\varphi} 
\DeclareMathOperator{\GL}{GL}
\DeclareMathOperator{\SL}{SL}
\DeclareMathOperator{\K}{K_1}
\DeclareMathOperator{\SK}{SK_1}
\DeclareMathOperator{\E}{E}
\DeclareMathOperator{\Mat}{Mat}
\DeclareMathOperator{\diag}{diag}
\DeclareMathOperator{\Int}{Int}
\DeclareMathOperator{\B}{B}
\DeclareMathOperator{\Bn}{\overline{\B_n(0)}}
\DeclareMathOperator{\Sp}{S} 
\DeclareMathOperator{\V}{V}
%\DeclareMathOperator{\E}{E}
% \newcommand{\exxp}[1]{\exp \left( #1 \right)}       
% \renewcommand{\sin}[1]{\sin \left( #1 \right)}      

%stuff, sometimes needed


%%% Local Variables:
%%% mode: latex
%%% TeX-master: "main"
%%% End:


\begin{document}

% pdftitle={Seminar: "K1 von Ringen"},             
% pdfauthor={Julian Seipel},                       
% pdfsubject={Seminararbeit},                      
% %TODO: abschluss-projekt von tex-kurs heranziehen


%% titlepage
\begin{titlepage}
  \begin{center}
  \centerline{\textbf{Simplizialkomplexe}}
  \vspace{1em}                    
  \centerline{\Large Seminar: Simpliziale Topologie}
  \vspace{1em}
  \centerline{\Large Julian Seipel}
  \vspace{0.5cm}
  \centerline{\today}
  \vspace{2cm}
\end{center}
\end{titlepage}


%\renewcommand{\theenumi}{\alpha{enumi}}
%\renewcommand{\thepage}{\Roman{\page}}
%\addtokomafont{part}{\itshape}
%\setkomafont{section}{\normalfont\Large}
%\setkomafont{subsection}{\normalfont\Large}
%\pagenumbering{Roman}

%\maketitle[99]

%inhaltsverzeichnis
\tableofcontents
\pagebreak
%gesamt content

% konventionen
%TODO: Schreibe die konventionen auf die in dieser ausarbeitung verwendet werden, zb. symbole für \R etc, was ist konvex , bla 

%kapitel über konventionen
\section*{Konventionen}

In diesem Seminarvortrag werden folgende Konventionen verwendet.
$\R$ steht für die reellen Zahlen. Eine Menge $A \subset \R^n$ ist konvex
wenn die Verbindungslinie je zweier Punkte vollkommen in $A$ enthalten ist.

Die konvexe Hülle einer Teilmenge eines Vektorraums, ist der Schnitt
aller konvexen Mengen die diese Mengen enthalten. Schreibe hierfür $\conv(A)$.

Es bezeichnet $e_i$ den Einheitsvektor in $\R^n$, wobei $e_0$ als der Nullvektor gesetzt wird.








%%% Local Variables:
%%% mode: latex
%%% TeX-master: "main"
%%% End:


%% kapitel über simplizies
% themenbereiche
%Geometrische simpliziale Komplexe; Triangulierungen; abstrakte
%simpliziale Komplexe; Beispiele simplizialer Komplexe

%NOTIZEN:
% warum simpliziale komplexe,mengen
% selbe homotophietheorie wie der ursprüngliche top raum

%warum simpliziale komplexe bzw mengen gebraucht werden, selbe homotophietheorie und selbe homotopiegruppen, einfachere berechnung der homotopiegruppen.

%homotopiegruppen sind invarianten von top räumen, einfaches kriterium
zur unterscheidung von top räumen


\begin{Def}[Geometrisch unabhängig]
	bla
\end{Def}

%Begriffe:
%geometrisch unabhängig




%% kapitel über geometrische simplizialkomplexe und simplizialabbildunge
% geometrische simplizialkomplexe

\section{Geometrische Simplizalkomplexe}

%TODO: bessere zusammenfassung schreiben
Die geometrischen Simpliziale allein reichen nicht aus um beliebige 
Teilmenge des $\R^N$ zu beschreiben bzw. modellieren, es ist noch notwendig eine Ansammlung von Simplizialen auf eine richtige art und Weise beschreiben zu können. Hierzu sind die Simplizialkomplexe da.

\begin{Def}[Geometrischer Simplizialkomplex]
	Wir nennen eine Menge $\Delta$ von (geometrischen) Simplizialen einen \textbf{geometrischen Simplizialkomplex}
	falls er folgende Bedingungen erfüllt:
	\begin{enumerate}[(K1)]
		\item $\emptyset \in \Delta$
		\item Für jeden Simplex $\sigma \in \Delta$ ist auch jede Seite
		$\tau$ in $\Delta$ enthalten
		\item Mit allen Simplexen $\sigma, \sigma' \in \Delta$ ist auch ihr
		Schnitt $\sigma \cap \sigma'$ in $\Delta$ enhalten.
	\end{enumerate}
\end{Def}

\begin{Bsp}
	\item Sei $\sigma^n \coloneqq  e_0 \ldots e_n $ der Standardsimplex.
		Dieser bildet mit all seinen Seiten einen geometrischen Simplizialkomplex, schreibe für die Menge aller Seiten von $\sigma$ hierfür $\Delta^n$.
%TODO: tikz bilder einfügen
\end{Bsp}

%TODO: polytope oder zugrunde liegender raum?
%\begin{Def}[Geometrische Realisierung]
%	Die Geometrische Realisierung eines Simplizalkomplexes $\Delta$ ist
%	die Vereinigung all seiner Elemente.
%	\begin{gather*}
%	|\Delta| \coloneqq \bigcup \left\{ \sigma \; \Big| \; \sigma \in
%	\Delta \right\} \subset \R^{\dim(\Delta)}
%	\end{gather*}
%\end{Def}

\begin{Def}[Unterkomplex, $k$-Skelett, Dimension, Polytop]
  Sei $\Delta$ ein Simplizialkomplex, so definiere
  \begin{enumerate}[1)]
  \item Eine Teilmenge $\Delta'$ von $\Delta$ heißt
    \textbf{Unterkomplex}, falls $\Delta'$ wiederum einen
    Simplizialkomplex bildet.
  \item Die Dimension eines Simplizalkomplexes ist wie folgt
    definiert:
    \begin{gather*}
      \dim(\Delta) \coloneqq \sup \left\{ \dim(\sigma) \; \Big| \;
        \sigma \in \Delta \right\}
    \end{gather*}
    Falls die Dimension unbeschränkt ist, wird diese auf unendlich
    gesetzt.
  \item Das \textbf{$k$-Skelett} $\Delta^{(k)}$ ist der Unterkomplex
    aller Simplizes mit der Dimension kleiner gleich $k$. Der
    Spezialfall $\Delta^{(0)}$ bezeichnet die Eckmenge von $\Delta$.
  \item Das \textbf{Polytop} ist wie folgt definiert
    \begin{gather*}
      |\Delta| \coloneqq \bigcup \left\{ \sigma \; \Big| \; \sigma \in
        \Delta \right\} \subset \R^{\dim(\Delta)}.
    \end{gather*}
    Dies entspricht dem Komplex zugrundeliegenden Raum.
	\end{enumerate}
\end{Def}



Definiere auf dem zugrunde liegenden Raum eines Simplizialkomplexes eine 
Topologie.

%TODO: zeige das dies topologie definiert
\begin{Def}[schwache Topologie]
	Sei $\Delta$ ein Simplizalkomplex, dann definiere die
	\textbf{schwache Topologie} auf dem Polytop $\gr{\Delta}$,
	durch folgende Charakterisierung der abgeschlossenen Mengen.
	\begin{gather*}
		A \subset \Delta \text{ abgeschlossen } :\Leftrightarrow A \cap
		\sigma \text{ abgeschlossen für alle } \sigma \in \Delta
	\end{gather*}
	Wobei auf der rechten Seite die Menge abgeschlossen bezüglich der
	Standardtopologie auf dem $\R^{\dim(\Delta)}$ ist.
\end{Def}

\begin{Lem}
  Die schwache Topologie ist feiner als die Standardtopologie und
  falls $\Delta$ endlich ist stimmen Standardtopologie, schwache
  Topologie auf $\gr{\Delta}$ überein.
  \begin{proof}
    Sei \OE\; $\gr{\Delta} \subset \R^N$ und $A \subset \gr{\Delta}$
    eine abgeschlossene Menge, so existiert ein $B \subset \R^N$
    abgeschlossen so dass $A = B \cap \gr{\Delta}$ gilt. Nun gilt für
    jeden Simplex $\sigma \in \Delta$ folgende Gleichheit:
    $A \cap \sigma = B \cap \gr{\Delta} \cap \sigma = B \cap \sigma$
    und somit ist $A \cap \sigma$ abgeschlossen bezüglich $\sigma$,
    also abgeschlossen bezüglich der schwachen Topologie.
		
    Falls nun $\Delta$ eine endliche Menge von Simplizes ist, gilt für
    eine abgeschlossene Menge $A$ bezüglich $\gr{\Delta}$ das
    $A \cap \sigma$ abgeschlossen in $\sigma$ ist und durch
    Vereinigung über die endlichen vielen Simplizes die
    Abgeschlossenheit bezüglich der Standardtopologie erreicht wird.
  \end{proof}
\end{Lem}

\begin{Bem}
  Die schwache Topologie ist im allgemeinen feiner als die
  Spurtopologie auf $\gr{\Delta}$ bezüglich dem $\R^N$. Betrachte
  hierzu den folgenden Simplizialkomplex,
  $\Delta \coloneqq \set{ \{t\} }{t \in \R}$.  Hierbei ist das
  Polytop gleich zu $\R$, aber die schwache
  Topologie entspricht der Diskreten. Für eine beliebige
  Teilmenge von $A \subset \R$ ist der Schnitt $A \cap \{ t \}$ mit
  einen beliebigen $0$-Simplex aus $\Delta$ gleich dem Simplex selbst oder die leere Menge,
  somit stets abgeschlossen, also ist jede Teilmenge abgeschlossen und
  damit die schwache Topologie auf dem Polytop der Diskreten und
  damit ungleich der Standardtopologie auf $\R$.
\end{Bem}

Beweise nun einige Aussagen über die schwache Topologie.

\begin{Satz}
  Sei $\Delta$ ein Simplizialkomplex, so gelten folgende Aussagen
  \begin{enumerate}[(1)]
        \item Die schwache Topologie ist hausdorffsch.
        \item Sei $A \subset \gr{\Delta}$ eine kompakte Teilmenge,
          dann existiert eine endlicher Unterkomplex $\Delta'$ der $A$
          enthält.
	\end{enumerate}
	\begin{proof}
          % TODO: noch zu beweisen
	\end{proof}
\end{Satz}

Definiere nun für spätere Zwecke Abbildung zwischen Simplizialkomplexen.

\begin{Def}[Simpliziale Abbildung]
	Seien zwei Simplizialkomplexe $\Delta, \Delta'$ gegeben, dann heißt
	eine Abbildung $f: \Delta^{(0)} \rightarrow \Delta'^{(0)}$ zwischen
	den Eckmengen eine \textbf{Simpliziale Abbildung}, falls für jeden
	Simplex $a_0 \ldots a_n$ aus $\Delta$ die Ecken auf einen Simplex
	$f(a_0) \ldots f(a_n)$ aus $\Delta'$ abgebildet werden. Man schreibt
	dann $f: \Delta \rightarrow \Delta'$.
\end{Def}

%TODO: durch sim abbildung induzierte abbildung, verkettung von induzierten ist wieder eine induzierte, 

\begin{Satz}
	Seien $\Delta, \Delta'$ zwei Simplizialkomplexe und $f: \Delta \rightarrow \Delta'$ eine simpliziale Abbildung, dann gelten
	folgende Aussagen
	\begin{enumerate}[(1)]
		\item $f(\Delta)$ ist ein Unterkomplex und das Urbild eines Unterkomplexes ist wieder ein Unterkomplex.
%TODO: ist das bild eines echten unterkomplexes wieder ein unterkomplex, siehe trivial gegenbeispiele
		\item Es gibt eine stetige Abbildung $\gr{f} : \gr{\Delta} \rightarrow \gr{\Delta'}$ die durch $f$ induziert wird und folgende Bedingung erfüllt:
			\begin{gather*}
				x = \sp{0}{n} t_i a_i \Rightarrow \gr{f}(x) = \sp{0}{n} t_i f(a_i) 
			\end{gather*}
			Man nennt $\gr{f}$ die induzierte simpliziale Abbildung.
		\item Die Verkettung von induzierten simplizialen Abbildung ist wieder eine induzierte Simpliziale.
		\item Falls $f$ eine bijektive simpliziale Abbildung ist, dann ist 
			die induzierte Abbildung $\gr{f}$ ein Homöomorphismus. Wir nennen dann $\gr{f}$ einen simplizialen Homöomorphismus, bzw. Isomorphismus.
		\item Sei $\Delta$ ein endlicher simplizialer Komplex, dann existiert eine natürliche Zahl $N$, so dass $\Delta$ isomorph zu 
		einem Unterkomplex von $\Delta^N$ ist.
	\end{enumerate}
	\begin{proof}
%TODO: noch zu bweisen, zu a) verwende coherent, 
foo	
	\end{proof}
\end{Satz}

\begin{Bsp}[Triangulation]
%TODO: beweise folgende triangulierengen \S^2 \simeq \gr{(\Delta^3)^{(2)}}, behauptung diese aussage gilt für allgemeines n, also \S^n \simeq \gr{(\Delta^{n+1})^{(n)}}
\end{Bsp}


%definiere (lineare) simpliziale abbildung, isomorphismus von komplexen
%und beweise das jeder endliche simplizialkomplex isomorphm zu einem unterkomplex
% 
% definire \Delta^n für n beliebige kardinalszahl, gilt die obige
% aussage immer noch wenn man unendliche komplexe betrachtet?


%TODO: ziegler - literatur : seite 11 (in reader abgelesen)



% \begin{Def}[Topologie auf dem Träger eines Simplizialkomplexes]
%   Eine Menge $U \subset \left| \Delta \right|$ ist genau dann offen,
%   falls für jeden Simplex $\sigma \in \Delta$ die Menge
%   $U \cap \sigma$ offen im unterliegenden Raum $\R^n$ ist.
% %	also bezüglich der spurtopologie auf \sigma
% % sie elementsOfalgtop seite 111 , review of quotient spaces, für andere beschreibung der topologie
% \end{Def}


%definitionen: unterkomplex, geometrische realisierung, k-skelett, schwache topologie, simpliziale abbildung, simplizialer iso, 

%aussagen: 	-	für endliche komplexe stimmen topologien überein
%			-	|\Delta| ist hausdorffsch
%			- 	Aussagen über kompaktheit, und kompakte teilmengen von \Delta
%			-	












%%% Local Variables:
%%% mode: latex
%%% TeX-master: "main"
%%% End:

%% kapitel über abstrakte simplizialkomplexe
%% abstrakte simplizialkomplexe

\section{Abstrakte Simplizialkomplexe}

Für topologische Anwendung ist der $\R^J$ in vielen Fällen zu konkret
und unnötiger gedanklicher Balast. Deshalb sollte man die
Simplizialkomplexe unabhängig vom euklidischen Raum definieren. Diese
abstrakten Simplizialkomplexe werden in diesem Abschnitt
behandelt. Die zunächst stark erscheinende Vereinfachung stellt sich
als eine äquivalente Darstellung zu den geometrischen Komplexen
heraus.

\begin{Def}
  Eine beliebige, nicht leere Menge $\Delta$ heißt \textbf{abstrakter
    Simplizialkomplex} oder auch nur abstrakter Komplex, falls mit
  jedem Element aus $\Delta$ auch jede Teilmenge aus diesem Element in
  $\Delta$ enthalten ist und jedes Element eine endliche
  Menge ist. Oder in Formeln geschrieben
%  \setcounter{equation}{(*)}
\renewcommand*{\theequation}{\textbullet}
  \begin{gather}
    \sigma \in \Delta \text{ und }  \tau \subset \sigma
    \Rightarrow \tau \in \Delta.
  \end{gather}
  Für einen abstrakten Komplex werden folgende Begriffe benötigt
  \begin{enumerate}[({A}1)]
% dimension, seite, knoten, unterkomplex, simpliziale abbildung, isomorphie
  \item Ein Element des Komplexes heißt \textbf{Seite}. Die Teilmenge aller
    Seiten der Mächtigkeit kleiner gleich $k$ heißt das \textbf{$k$-Skelett},
    insbesondere für $k=1$ ist dies die \textbf{Knotenmenge}.
  \item Die \textbf{Dimension} des Komplexes ist das Supremum über die
    Mächtigkeit der Seiten von $\Delta$.
  \item Eine Teilmenge vom Komplex, die (\textbullet) erfüllt, heißt
    \textbf{Unterkomplex}.
  \item Eine Abbildung zwischen den Knotenmenge zweier abstrakten
    Komplexe heißt \textbf{simpliziale Abbildung}, falls Seiten auf
    Seiten abgebildet werden.
  \item Zwei abstrakte Komplexe heißen \textbf{isomorph}, falls es eine
    bijektive simpliziale Abbildung gibt, deren Inverses auch
    simplizial ist.
  \item Sei $\Delta$ ein geometrischer Komplex, so bezeichnet die Menge
    \begin{gather*}
      V(\Delta) \coloneqq \set{ \sset{a_0 \ldots a_n} \subset
        \Delta^{(0)} }{ a_0 \ldots a_n \in \Delta}
    \end{gather*}
    das \textbf{Knotenschema} von $\Delta$.
  \item Ist ein abstrakter Komplex $\Delta$ isomorph zum Knotenschema
    eine geometrischen Komplexes $\Delta'$, so nennen wir $\Delta'$
    eine \textbf{geometrische Realisierung} von $\Delta$.
  \end{enumerate}
\end{Def}

\begin{Bsp}
  \begin{enumerate}[\textbullet]
  \item Für einen geometrischen Komplex $\sigma$ ist die Menge aller
    Knotenschemata zu jeder Seite von $\sigma$ ein abstrakter
    Simplizalkomplex und das Knotenschema wird auf triviale Art und
    Weise geometrisch durch $\sigma$ realisiert.
  \item Die Menge $\Delta = \sset{%
      \sset{a,b},\sset{a,c},%
      \sset{b,c},\sset{a},\sset{b},\sset{c},\emptyset}$ ist ein Beispiel und 
    seine geometrische Realisierung sieht wie folgt aus.
    \begin{center}
      \begin{tikzpicture}
        \draw [thin] (90:1) to (210:1) to (330:1) to (90:1);
        \foreach \foo in {90,210,330}
        \fill (\foo:1) circle[radius=0.07cm];

        \node (a) at (90:1.3) {$a$};
        \node (b) at (210:1.3) {$b$};
        \node (c) at (330:1.3) {$c$};
      \end{tikzpicture}
    \end{center}

  \item Betrachte die Menge
    $\set{ \sset{i,i+1} , \sset{i} }{ i \in \Z} \cup \sset{\emptyset}$,
    dann ist der Polyeder zu einer geometrischen Realisierung
    homöomorph zu $\R$.

  \item Für eine Menge $M$, deren Elemente endliche Mengen sind, lässt
    sich stets ein abstrakter Komplex bilden, indem man alle
    Teilmengen der Elemente hinzu nimmt. Schreibe dies als
    $\pow^*(M)$.

  \item Die von-Neumann Zahlen liefern als Menge einen nicht endlichen
    abstrakten Komplex.  Diese Menge erhält man wie folgt. Definiere
    rekursiv $V_0 \coloneqq \emptyset$ und
    $V_{n+1} \coloneqq \sset{V_n} \cup \V_n$, dann ist
    $V \coloneqq \bigcup\limits_{n \geq 0} V_n$ ein nicht endlicher
    Komplex.
  \item Ein ungerichteter Graph $G=(V,K)$ mit $V$ die Eckmenge und $K$
    die Kanten ist ein abstrakter Simplizialkomplex.
  \item Das Knotenschema von einem geometrischen Komplex ist ein
    abstrakter Komplex.
  \end{enumerate}
\end{Bsp}

Der folgende Satz hat als Aussage, dass das letzte Beispiel das
Entscheidende ist. Der Zusammenhang zwischen geometrischen und
abstrakten Komplexen ergibt sich aus dem folgenden Satz.


\begin{Satz}
  Es gelten folgende Aussagen
  \begin{enumerate}[(1)]
  \item Für jeden abstrakten Komplex existiert ein geometrischer
    Komplex, dessen Knotenschema (abstrakt) isomorph zu diesem Komplex
    ist.
  \item Zwei geometrische Komplexe sind (linear) isomorph, genau dann
    wenn die dazuentsprechenden Knotenschemata (abstrakt) isomorph
    sind.
  \end{enumerate}
  \begin{proof}
    \begin{enumerate}[(1)]
    \item Nach \cref{satz:geosimp} $(e)$ existiert für einen
      endlichen geometrischen Komplex stets ein Standardkomplex
      $\Delta^n$, sodass dieser als ein isomorpher Teilkomplex darin
      enthalten ist. Verallgemeinere nun diesen Standardkomplex auf
      einen unendlich dimensionalen im Raum $\E^J$. Hierzu wird
      $\Delta^J$ von all den Einheitsvektoren im $\E^J$
      aufgespannt. Dies bildet wiederrum einen geometrischen Komplex.

      Sei nun ein abstrakter Komplex $\Delta$ und seine Knotenmenge
      $\Delta^{(0)}$ gegeben, dann betrachte die Einbettung
      $f : \Delta^{(0)} \rightarrow J$.  Es ist nun klar, dass die
      Simplexe, die von den Einheitsvektoren $e_j$ aus dem Bild von
      $f$ aufgespannt werden, einen geometrischen Komplex bilden, der
      nach Konstruktion ein Knotenschema besitzt, das isomorph zu
      $\Delta$ ist.

    \item Seien zwei geometrischen Komplexe $\Delta,\Delta'$ mit ihren
      Knotenschemata $V(\Delta),V(\Delta')$ gegeben. Falls nun ein
      Isomorphismus zwischen $\Delta,\Delta'$ existiert dann erfüllt
      er eingeschränkt auf die Knotenmenge genau die Eigenschaft der
      abstrakten simplizialen Abbildung zwischen den Knotenschemata
      als abstrakte Komplexe und somit entsprechen sich diese
      Isomorphismen gegenseitig.
    \end{enumerate}
  \end{proof}
\end{Satz}

% todo: topologie --literatur/archiv seite 333ff

% erkläre das labeln von triangulierungen, bzw
% geometrischen,abstrakten simplizialkomplexen

\begin{Bem}[Label,Abwicklung]
  Da im weiteren Verlauf des Seminars meist nicht zweidimensionale
  Triangulierungen angegeben und benötigt werden, ist es notwendig,
  eine vereinfachte Darstellung dieser höher dimensionalen Objekte
  abzugeben. Hierzu wird das Labeln von Triangulierungen angegeben.

  Betrachte zunächst für die Idee des Verfahrens den Tetraeder und
  seine Standardabwicklung im Zweidimensionalen.
  % zeichne dreidim tetraeder, und eine zweidim abwicklung, tikz
  Bei der Abwicklung werden die $0$-Simplexe fortlaufend alphabetisch
  nummeriert. Bei Knoten, die in der ursprünglichen Dimension
  zusammenfallen wird das gleiche Label verwendet.

%TODO: auch für unendlich dimensionale abwicklung
  Die Formalisierung der obigen Idee wird durch die Angabe eines
  abstrakten Komplexes gegeben, dessen geometrische Realisierung dem
  geometrischen Komplex entspricht. Sei $\Delta$ ein endlich
  dimensionaler geometrischer Komplex gegeben. Dann ist die Abwicklung
  durch einen zweidimensionalen geometrischen Komplex $\Delta'$ und
  eine surjektive Abbildung $f: \Delta \rightarrow \Delta'$ gegeben.

%TODO: weiter ausführen

\end{Bem}

% TODO: label und zweidim abwicklungen vom torus, zylinder, kugel und kleinsche flasche

%einfache beispiele

\begin{Bsp}[Triangulation]
  Es werden nun Triangulierungen für Kugel, Zylinder und Torus
  angegeben durch abstrakte Komplexe deren Geometrische Realisierungen
  eine Triangulation der entsprechenden Flächen im $\R^3$ sind.

  \begin{description}
  \item[Zylinder:] Betrachte die Menge
    \begin{gather*}
      \Delta \coloneqq
      \sset{\sset{a,b,c},\sset{b,c,e},\sset{c,e,f},\sset{c,d,f}%
        ,\sset{a,d,f},\sset{a,b,f}}
    \end{gather*}
    Dann ist $\pow^*(\Delta)$ ein abstrakter Komplex.  Durch
    identifizieren der Punkte gleichen Labels wird durch
    zusammenfalten im dreidimensionalen daraus eine Fläche, die
    homöomorph zur Zylinderoberfläche ist. Dies sieht man unmittelbar
    ein, indem man zunächst annimmt, dass die Kante $\sset{a,b}$
    parallel zur $z$-Achse im $\R^3$ ist und dann eine
    Normierungsabbildung in der $x-y$-Ebene als Abbildung verwendet.
    \newline
  \begin{center}
      \parbox{0.7\linewidth}{%
      \begin{tikzpicture}
        \draw(0,0) rectangle (3,1); 
        \draw (1,0) to (1,1);
        \draw (2,0) to (2,1);
        \draw (0,1) to (1,0) to (2,1) to (3,0);

        \node (a1) at (0,-0.2) {$a$};
        \node (c) at (1,-0.2) {$c$};
        \node (d) at (2,-0.2) {$d$};
        \node (a2) at (3,-0.2) {$a$};

        \node (b1) at (0,1.2) {$b$};
        \node (e) at (1,1.2) {$e$};
        \node (f) at (2,1.2) {$f$};
        \node (b2) at (3,1.2) {$b$};
      \end{tikzpicture}
      \hfill
      \begin{tikzpicture}
        \node (a) at (-0.2,0.2) {$a$};
        \node (b) at (-0.2,2.2) {$b$};
        \node (c) at (2,0.7) {$c$};
        \node (d) at (1.8,0) {$d$};
        \node (e) at (2,2.7) {$e$};
        \node (f) at (1.5,2.2) {$f$};
%        \draw [thin,fill=lightgray] (0,0.2) to (1.8,0) to (1,1.7) to (0,0.2);
%        \draw [thin,fill=lightgray] (1.8,0) to (1.8,0.7) to (1,1.7) to (1.8,0);
%        \draw [thin,dashed] (1.8,0.7) to (0,0.2);
        \draw [thin] (1.6,0) to (1.8,0.7) to (0,0.2) to (1.6,0);
        \draw [thin] (1.6,2) to (1.8,2.7) to (0,2.2) to (1.6,2);
        \draw [thin] (1.6,0) to (1.8,0.7) to (1.8,2.7) to (1.6,2) to (1.6,0);
        \draw [thin] (0,0.2) to (0,2.2);
        \draw [thin,dashed] (0,2.2) to (1.8,0.7);
        \draw [thin] (1.8,0.7) to (1.6,2);
        \draw [thin] (1.6,2) to (0,0.2);
      \end{tikzpicture}}
    \end{center}
  \item[Torus:] Zur Triangulierung des Torus ($\Sp^1 \times \Sp^1$)
    ist ein wesentlich komplizierterer abstrakter Komplex
    vonnöten.

    Betrachte hierzu die Menge
    \begin{align*}
      \Delta &\coloneqq \{ \sset{a,c,i},\sset{a,d,i},\\
        &\sset{b,c,f},\sset{c,f,i},\sset{a,b,d},\sset{b,d,f},\\
        &\sset{d,h,i},\sset{d,e,h},\sset{i,g,h},\sset{i,g,f},\\
        &\sset{d,e,f},\sset{e,f,g},\sset{a,c,e},\sset{c,e,h},\\
        &\sset{b,c,h},\sset{b,g,h},\sset{a,b,g},\sset{a,e,g} \}
    \end{align*}

    Dann ist $\pow^*(\Delta)$ ein abstrakter Komplex.

    \begin{center}
    \parbox{0.8\linewidth}{%
      \begin{tikzpicture}
      \draw[step=1] (0,0)grid(3,3);    
      \draw [thin] (0,3) to (3,0);
      \draw [thin] (0,2) to (2,0);
      \draw [thin] (0,1) to (1,0);
      \draw [thin] (1,3) to (3,1);
      \draw [thin] (2,3) to (3,2);

        \node (a1) at (-0.2,-0.2) {$a$};
        \node (a2) at (3.2,-0.2) {$a$};
        \node (a3) at (3.2,3.2) {$a$};
        \node (a4) at (-0.2,3.2) {$a$};

        \node (b1) at (-0.2,1) {$b$};
        \node (b2) at (3.2,1) {$b$};
        \node (c1) at (-0.2,2) {$c$};
        \node (c2) at (3.2,2) {$c$};

        \node (d1) at (1,-0.2) {$d$};
        \node (d2) at (1,3.2) {$d$};
        \node (e1) at (2,-0.2) {$e$};
        \node (e2) at (2,3.2) {$e$};

        \node (f) at (0.8,0.75) {$f$};
        \node (g) at (1.8,0.8) {$g$};
        \node (h) at (1.8,1.8) {$h$};
        \node (i) at (0.8,1.8) {$i$};

    \end{tikzpicture}
    \hfill
    \begin{tikzpicture}[scale=1.3]
      \draw ellipse (2cm and 1cm);
      \begin{scope}
        \clip (-0.7,0)rectangle(0.7,0.5);
        \draw ellipse (0.7cm and 0.25cm);
      \end{scope}
      \begin{scope}
        \clip (-0.9,0.1)rectangle(0.9,-0.5);
        \draw (0,0.3) ellipse (1cm and 0.4cm);
      \end{scope}
    \end{tikzpicture}}
  \end{center}
  Identifiziert man nun jeweils die gegenüberliegende Seiten, dann
  erhält man den Torus als zwei dimensionale Fläche im $\R^3$.
\item[Kugel:]
  Betrachte den Komplex
  \begin{gather*}
    \Delta \coloneqq \{ \sset{a,b,c},\sset{a,c,d},
    \sset{a,b,d},\sset{b,c,d} \}
  \end{gather*}
  und alle Teilmengen der Elemente oder kürzer geschrieben durch
  $\pow(\sset{a,b,c,d}) \setminus \{ a,b,c,d \}$.  Dann ist hierdurch
  eine Triangulierung der $\Sp^2$ gegeben.

  \begin{center}
    \parbox{0.7\linewidth}{%
\begin{tikzpicture}
  \node (a1) at (30:2.2) {$a$};
   \node (a2) at (150:2.2) {$a$};
   \node (a3) at (270:2.2) {$a$};
   \node (b) at (90:1.2) {$b$};
   \node (c) at (210:1.2) {$c$};
   \node (e) at (330:1.2) {$d$};

%  \draw [thin] (90:1) to (210:1) to (330:1) to (90:1);
  \draw [thin] (330:1) to (30:2) to (90:1) to (330:1);
  \draw [thin] (90:1) to (150:2) to (210:1) to (90:1);
  \draw [thin] (210:1) to (270:2) to (330:1) to (210:1);
\end{tikzpicture}
\hfill
\begin{tikzpicture}[scale=1.7]
  \draw [thin] (0,1.2) to (1.8,1) to (1,2.7) to (0,1.2);
  \draw [thin] (1.8,1) to (1.8,1.7) to (1,2.7) to (1.8,1);
  \draw [thin,dashed] (1.8,1.7) to (0,1.2);
\end{tikzpicture}}
\end{center}

Beschreibt man nun diesen erhaltenen Tetraeder in eine Sphäre ein, so
folgt genauso wie in \cref{bsp:triangulierung}, dass dies eine
Triangulierung der Sphäre liefert.

\end{description}
  
\end{Bsp}

%TODO: beweise folgende triangulierengen \S^2 \simeq \gr{(\Delta^3)^{(2)}}, behauptung diese aussage gilt für allgemeines n, also \S^n \simeq \gr{(\Delta^{n+1})^{(n)}}





% abzählbare mengen können nur durch 0-simplexe trianguliert werden



%Begriffe:
%geometrisch unabhängig, und linear unabhängig vergleichen gegenüberstellen
% simplex, geometrischer simplizialkomplex, abstrakter simplizialkomplex 


% n-ebene, menge von punkten, aufgespannt durch geo.unab system mit konvexkombinationen
% die faktoren sind eindeutig bestimmt

%affine transformation: x -> Ax+b

%geo.unab systeme werden durch affine transformationen auf geo.unab systeme abgebildet

% n simplex

% baryzentrische koordinaten

% ist die leere menge eine simplizial?

% ein simplex ist genau die konvexe hülle von einer endlichen teilmenge vom \R^N

% eine seite eines simplex ist ein teilsimplex der dimension d

% anzahl der d teilsimplexe ist binomial_koeff(n+1,d+1), für einen n dim simplex
% klar wähle aus den n+1 punkten d+1 stück aus

% verwende das schläfli symbol zur beschreibung der umgebung eines
% punktes aus einem simplex für triangulierungen werden nur 2fache
% schläflisymbole benötigt, also {p,q}, da bei der triangulierung nur
% \laplace^2 simplexe verwendet werden, vereinfacht sich das symbol
% auf die form {3,q}. gebe nun zu jedem punkt auf der
% mannigfaltigkeit, also den 0 dimensionalen simplexen, die anzahl der
% angrenzenden dreiecker/laplace^2 simplexe an


% schreibe in die appendix ein eigenen anhang nur mit getikzten beispielen
% die komplette pflasterung von \R^2 oder allgemein die füllung des \R^n mit \laplace^n
% verschiedene triangulierungen

% zeige nicht-kompakte teilmengen vom \R^n sind nicht triangulierbar,
% einfachstes bsp (0,1), einfach das geometrische realisierungen stets
% kompakte mengen sind, somit kann es keinen homöomorphismus geben


%%% Local Variables:
%%% mode: latex
%%% TeX-master: "main"
%%% End:
%\printindex

%\index{marker}
%\nocite{*}

%TODO: Literatur


%TODO: stichwortverzeichnis mit makeindex setzen
%\printbibliography

\end{document}
