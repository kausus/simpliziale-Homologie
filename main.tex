\documentclass[paper=a4, twoside=false, twocolumn=false, ngerman, fontsize=10pt, titlepage=false, parskip=half, abstract=false,toc=flat,left,listof,indent]{scrartcl}

%\makeindex

%TODO: vll den Seitenkopf/fuss noch bearbeiten
%\pagestyle{scrheadings}
%\ofoot[{\includegraphics[height=10ex]{unilogo.pdf}}]{\includegraphics[height=10ex]{unilogo.pdf}}

%% encodings                                        
\usepackage[T1]{fontenc}                            
\usepackage[utf8]{inputenc}                         
\usepackage[shorthands=off]{babel}                  
\usepackage{lmodern}                                
\usepackage{microtype}                              
\usepackage{scrlayer-scrpage}                       
\usepackage{enumerate}                              
%\usepackage{enumitem}
\usepackage{csquotes}                               
%\usepackage[backend=biber,style=numeric-comp]{biblatex}
%\bibliography{bib.bib}

% %maths                                            
\usepackage{amsmath, amsthm, amssymb, bm, bbm, mathtools, dsfont, mathrsfs}

%zum spielen
\usepackage{verbatim}

%commutative diagramms
%\usepackage{tikz-cd}

% links, TODO
%\usepackage[pdftitle={Seminar: "K1 von Ringen"},pdfauthor={Julian Seipel}, pdfsubject={Seminararbeit}]{hyperref}
%TODO: abschluss-projekt von tex-kurs heranziehen

%\renewenvironment{proof}{{\bfseries Proof}}{*something*}
% \newenvironment{myproof}[1][\proofname]{%
%   \proof[\ttfamily \scshape \large #1 (yes, ``#1'')]%
% }{\endproof}

                                                    
%special mathsymbols                                
\newcommand{\C}{\mathds{C}}                         
\newcommand{\R}{\mathds{R}}                         
\newcommand{\N}{\mathds{N}}                         
\newcommand{\Z}{\mathds{Z}}                         
\newcommand{\Q}{\mathds{Q}}
\newcommand{\pma}[1]{\begin{pmatrix} #1 \end{pmatrix}}
\newcommand{\A}{\mathscr{A}}
\newcommand{\F}{\mathbb{F}}                                            
\newcommand{\gs}{$\{a_0\ldots a_n\}$ }
\newcommand{\gr}[1]{\left| #1 \right|}
% TODO: schreibe newcommand, das #2 als optional nimmt, und somit kein
% \Big| setzt
\newcommand{\set}[2]{\left\{ #1 \; \Big| \; #2 \right\}}
%\newcommand{\dim}{Dim}            
%\newcommand{\sup}{sup}
       


%\newcommand{\SS}{\mathbb{S}}
                    
%%%%%%%%%%%%%%%%%%%%                                
%theoremstyle                                       
%\theoremstyle{definition}                           
%\newtheorem*{Def}{Defintion}                        
%\theoremstyle{plain}                               
%\newtheorem{Beh}{Beh}                              
%\newtheorem{Vor}{Vor}                                                    
                          
% math (theorems)                     
\theoremstyle{definition}             
\newtheorem{Def}{Definition}[section] 
\newtheorem{Bsp}[Def]{Beispiel}       

\theoremstyle{remark}               
\newtheorem{Bem}[Def]{Bemerkung}      
\newtheorem{Wdh}[Def]{Wiederholung}   

\theoremstyle{plain}                  
\newtheorem{Satz}[Def]{Satz}          
\newtheorem*{satz}{Satz}              
\newtheorem*{Beh}{Behauptung}
\newtheorem{Lem}[Def]{Lemma}          
\newtheorem{Kor}[Def]{Korollar}       
\newtheorem{Fol}[Def]{Folgerung}      

%TODO: schrift soll geändert werden,
%\renewenvironment*{proof}{\bfseries}{}

%stuff used only a few times                        
\DeclareMathOperator{\f}{\varphi} 
\DeclareMathOperator{\GL}{GL}
\DeclareMathOperator{\SL}{SL}
\DeclareMathOperator{\K}{K_1}
\DeclareMathOperator{\SK}{SK_1}
\DeclareMathOperator{\E}{E}
\DeclareMathOperator{\Mat}{Mat}
\DeclareMathOperator{\diag}{diag}
\DeclareMathOperator{\Int}{Int}
\DeclareMathOperator{\B}{B}
\DeclareMathOperator{\Bn}{\overline{\B_n(0)}}
\DeclareMathOperator{\Sp}{S} 
% \newcommand{\exxp}[1]{\exp \left( #1 \right)}       
% \renewcommand{\sin}[1]{\sin \left( #1 \right)}      

%stuff, sometimes needed


%%% Local Variables:
%%% mode: latex
%%% TeX-master: "main"
%%% End:


\begin{document}

% pdftitle={Seminar: "K1 von Ringen"},             
% pdfauthor={Julian Seipel},                       
% pdfsubject={Seminararbeit},                      
% %TODO: abschluss-projekt von tex-kurs heranziehen


%% titlepage
\begin{titlepage}
  \begin{center}
  \centerline{\textbf{Simplizialkomplexe}}
  \vspace{1em}                    
  \centerline{\Large Seminar: Simpliziale Topologie}
  \vspace{1em}
  \centerline{\Large Julian Seipel}
  \vspace{0.5cm}
  \centerline{\today}
  \vspace{2cm}
\end{center}
\end{titlepage}


%\renewcommand{\theenumi}{\alpha{enumi}}
%\renewcommand{\thepage}{\Roman{\page}}
%\addtokomafont{part}{\itshape}
%\setkomafont{section}{\normalfont\Large}
%\setkomafont{subsection}{\normalfont\Large}
%\pagenumbering{Roman}

%\maketitle[99]

%inhaltsverzeichnis
\tableofcontents
\pagebreak
%gesamt content

% konventionen
%TODO: Schreibe die konventionen auf die in dieser ausarbeitung verwendet werden, zb. symbole für \R etc, was ist konvex , bla 

%kapitel über konventionen
\section*{Konventionen}

In diesem Seminarvortrag werden folgende Konventionen verwendet.
$\R$ steht für die reellen Zahlen. Eine Menge $A \subset \R^n$ ist konvex
wenn die Verbindungslinie je zweier Punkte vollkommen in $A$ enthalten ist.










%%% Local Variables:
%%% mode: latex
%%% TeX-master: "main"
%%% End:


%% kapitel über simplizies
% themenbereiche
%Geometrische simpliziale Komplexe; Triangulierungen; abstrakte
%simpliziale Komplexe; Beispiele simplizialer Komplexe


%homotopiegruppen sind invarianten von top räumen, einfaches kriterium
%zur unterscheidung von top räumen


% schlagwort verzeichnis
%TODO: folgende begriffe müssen definiert, verwendet und verstanden sein: simplex, seite, eckmenge, dimension, simplizialkomplex, n-skelett, geometrischer simplizialkomplex, geometrische realisierung, schwache topologie, triangulierbar, simpliziale abbildung, baryzentrische koordinaten, 

%warum simpliziale komplexe bzw mengen gebraucht werden, selbe homotophietheorie und selbe homotopiegruppen, einfachere berechnung der homotopiegruppen.

\section{Geometrische Simpliziale}


Topologische Räume sind im allgemeinen sehr schwer zu beschreibende
mathematische Objekte. Dieses Seminar behandelt eine Methode, eine
bestimmte Klasse von topologischen Räumen durch einfache geometrische
Objekte, den Simplizialen, zu beschreiben und zu verstehen.

Simpliziale sind einfache geometrische Objekte für beliebige
Dimensionen. Diese bilden eine unmittelbare Verallgemeinerung der
bekannten in den Dimension kleiner gleich drei, wie das gleichseitige
Dreieck und den Tetraeder.

Wir definieren zunächst grundlegende Begriffe für die anschließende
Defintion der Simpliziale.

\begin{Def}[Geometrisch unabhängig\footnote{Oder auch affine
    Unabhängigkeit}]
  \label{def:1}
  Eine endliche Menge $\{ a_0,\ldots ,a_n \} \subset \R^N$ heißt
  \textbf{geometrisch unabhängig}, falls das System von Vektoren
  \begin{gather*}
    a_0 - a_1 , a_0 - a_2 \ldots , a_0 - a_n
  \end{gather*}
  unabhängig im Sinne der Linearen Algebra ist. Eine beliebige Menge
  $A \subset \R^J$ heißt geometrisch unabhängig, falls jede endliche
  Teilmenge von $A$ geometrisch unabhängig im obigen Sinne ist.
\end{Def}
Zeige für spätere Verwendung eine äquivalente Formulierung der
geometrischen Unabhängigkeit.

\begin{Lem}\label{lem:1}
  Teilsysteme von geometrisch unabhängigen Systemen sind geometrisch
  unabhängig. Eine endliche, geometrisch unabhängige Menge des $\R^N$
  hat maximal $N+1$ Elemente und für eine Menge
  $\{ a_0 , \ldots , a_n \} \subset \R^N$ sind folgende Aussagen
  äquivalent:

  \begin{enumerate}[(i)]
  \item $\{ a_0 , \ldots , a_n \}$ ist geometrisch unabhängig.
  \item Für $\sum\limits_{i=0}^n t_i = 0$ und
    $\sum\limits_{i=0}^n t_i a_i = 0$ folgt stets $t_i = 0$ für alle
    $i \in \{ 0,\ldots,n\}$.
  \end{enumerate}

  \begin{proof}
    Sei $\{ a_0 , \ldots , a_n \}$ ein geometrisch unabhängiges System
    und $\{ a_{i_0},\ldots,a_{i_r} \}$ hierzu ein Teilsystem.  Sei
    \OE~ $i_0 = 0$, sonst nummeriere um oder betrachte ein anderes
    $i_j$, so dass für dieses $i_j$ das ursprüngliche geometrisch
    unabhängige System der Punkt $a_{i_j}$ als Basispunkt gewählt
    werden kann. Nun ist nach \cref{def:1},
    $ a_0 - a_1 , \ldots , a_0 - a_n$ linear unabhängig und somit auch
    das Teilsystem $ a_0 - a_ {i_1}, \ldots , a_0 - a_{i_r}$.  Sei nun
    $A \subset \R^N$ ein endliches, geometrisch unabhängiges System,
    dann ist nach \cref{def:1}, für ein Element $a \in A$, das System
    $\set{ a - a'}{a' \in A \setminus \sset{a} }$ linear unabhängig.
    Die Kardinalität des Systems ist offentsichtlich durch die
    Dimension des Vektorraum beschränkt und somit maximal gleich $N$,
    mit dem Basispunkt also $N+1$. Für den letzten Teil des Beweises
    nutze man folgende Äquivalenzen

    \begin{description}
    \item[i) $\Rightarrow$ ii)] Seien $\sp{0}{n} t_i = 0$ und $\sp{0}{n} t_i
      a_i = 0$, dann folgt 
      \begin{gather*}
        0 = \sp{0}{n} t_i a_i = t_0 a_0 + \sp{1}{n} t_i a_i = -a_0
        \sp{1}{n} t_i + \sp{1}{n} t_i a_i = \sp{1}{n} t_i (a_i - a_0)
    \end{gather*}
    und somit $t_i = 0$ für alle $i$.
      \item[ii) $\Rightarrow$ i)] Betrachte folgende Gleichungen
        \begin{gather*}
          0 = \sp{1}{n} t_i (a_i - a_0) = -a_0 \cdot \sp{1}{n} t_i 
          + \sp{1}{n} t_i a_i = \sp{0}{n} t_i a_i.
        \end{gather*}
        Mit $t_0 \coloneqq - \sp{1}{n} t_i$ folgt nun mit ii) die
        Behauptung.
        % Sei nun $\sp{1}{n} t_i (a_i - a_0) = 0$ 
    \end{description}\nopagebreak
  \end{proof}
\end{Lem}

In der nächsten Definition wird das in diesem Vortrag zugrundeliegende
Objekt von Interesse definiert. Dieses dient im weiteren Verlauf als
Baustein für die Komplexe.

\begin{Def}[Geometrischer $n$-Simplex]
  Zu einem geometrisch unabhängigen System \gs $\subset \R^N$, nennt
  man die Menge
  \begin{gather*}
    \set{\sum\limits_{i=0}^n t_i a_i \in \R^N}{ t_i \in [0,1] ~,~
      \sum\limits_{i=0}^n t_i = 1}
  \end{gather*}
  den (geometrischen) $n$-Simplex und schreibt $a_0 \ldots a_n$ oder
  ohne die Punkte genauer zu spezifizieren $\sigma^n$. Dies ist die
  Menge aller Konvexkombinationen des Systems \\\gs. Als Konvention ist
  der Simplex stets in den $\R^N$ eingebettet für ein $N \geq n$.
\end{Def}


\begin{Bsp}[Einfache Beispiele]
  Definiere für $n \in \N$ den \textbf{Standardsimplex}
  $\sigma^n \coloneqq e_0\ldots e_n$. Die Fälle $n = 0,1,2,3$ sind
  hier aufgezeichnet. \newline

% 0 - simplex, punkt
% besseren punkt ausdenken, kleinen kreis mit blauer farbe, 
\begin{tikzpicture}
  \node (s) at (0,2) {$n=0$};
  \fill (0,0) circle[radius=0.07cm];
\end{tikzpicture}
\hfill
%1 - simplex, gerade
\begin{tikzpicture}
 \node (s) at (0,2) {$n=1$};
  \fill (-0.7,0) circle[radius=0.07cm];
%  \draw[thin] (0,0)--++(1,0);
  \fill (0.7,0) circle[radius=0.07cm];

  \draw [thin,fill=lightgray] (-0.7,0) to (0.7,0) to (-0.7,0);
\end{tikzpicture}
\hfill
%2 - simplex, gleichseitiges dreieck
%PROBLEM: drucker druckt nicht den fill
\begin{tikzpicture}
    \node (s) at (0,1.5) {$n=2$};
  \draw [thin,fill=lightgray] (90:1) to (210:1) to (330:1) to (90:1);
    \fill (90:1) circle[radius=0.07cm];
    \fill (210:1) circle[radius=0.07cm];
    \fill (330:1) circle[radius=0.07cm];
\end{tikzpicture}
\hfill
%3 - simplex, tetraeder
\begin{tikzpicture}
  \node (s) at (1,2) {$n=3$};
\draw [thin,fill=lightgray] (0,0.2) to (1.8,0) to (1,1.7) to (0,0.2);
\draw [thin,fill=lightgray] (1.8,0) to (1.8,0.7) to (1,1.7) to (1.8,0);
\draw [thin,dashed] (1.8,0.7) to (0,0.2);

    \fill (0,0.2) circle[radius=0.07cm];
    \fill (1.8,0) circle[radius=0.07cm];
    \fill (1,1.7) circle[radius=0.07cm];
    \fill (1.8,0.7) circle[radius=0.07cm];
\end{tikzpicture}

\end{Bsp}

Um Punkte aus einem Simplex unabhängig von der Lage des Simplex im
umliegenden Raum zu beschreiben, sind spezielle Koordinaten
vonnöten. Diese werden nun definiert und ihre Eindeutigkeit und
Stetigkeit bezüglich des Punktes bewiesen.


\begin{Lem}[Baryzentrische Koordinaten]\label{lem:bary}
  \normalfont Es bezeichnet $x$ einen Punkt aus dem Simplex
  $\sigma^n = a_0\ldots a_n$. In der Darstellung
  $x = \sum_{i=0}^n t_i a_i$ nennt man $t_i$ \textbf{baryzentrische
    Koordinaten}. Diese sind durch $x$ eindeutig bestimmt und als
  Funktionen $t_i : \sigma^n \subset \R^N \rightarrow [0,1]$ stetig.
  \pagebreak
  \begin{proof}
    Zeige zunächst die Eindeutigkeit über die äquivalente
    Formulierung der geometrischen Unabhängigkeit aus \cref{lem:1}
    \begin{description}
    \item[Eindeutigkeit: ] Seien zwei Darstellungen
      \begin{gather*}
        x = \sum\limits_{i=0}^n t_i a_i = \sum\limits_{i=0}^n s_i a_i
        \text{ mit } \sum\limits_{i=0}^n t_i = \sum\limits_{i=0}^n s_i
        = 1
      \end{gather*}
      gegeben, so folgt durch umformen
      \begin{gather*}
        \sum\limits_{i=0}^n (t_i - s_i ) \cdot a_i = 0 \text{ und }
        \sum\limits_{i=0}^n t_i - s_i = 0.
      \end{gather*}
      Da $\sset{a_0 , \ldots , a_n}$ ein geometrisch unabhängiges
      System ist, folgt durch \cref{lem:1}, dass
      $t_i - s_i = 0 \Leftrightarrow t_i = s_i$, also die Koordinaten
      eindeutig sind.
    \item[Stetigkeit: ] Aufgrund der Eindeutigkeit der baryzentrischen
      Koordinaten sind die Abbildungen $t_i (x)$ wohldefiniert.
      Definiere die Vektoren $b_i \coloneqq a_i - a_0$ und erweitere
      das linear unabhängige System $\set{ b_i }{ 1 \leq i \leq n}$ zu
      einer Basis des $\R^N$. Schreibe hierfür $\set{ b_i
      }{ 0 \leq i \leq N}$. Betrachte nun die Differenz $x - a_0$
%	TODO: auf diegleichung mit cref referenzieren
      \begin{align*}
        x - a_0 &= \sp{0}{n} t_i a_i - 1 \cdot a_0\\
                &\overset{(*)}{=} \sp{0}{n} t_i a_i - \sp{0}{n} t_i a_0\\
                &= \sp{1}{n} t_i \cdot (a_i - a_0)\\
                &= \sp{1}{n} t_i b_i\\
                &= \sp{1}{n} t_i b_i + \sp{n+1}{N} 0 \cdot b_i.
      \end{align*}
      Hierbei wird $(*):$ $\sp{0}{n} t_i = 1$ verwendet. Schreibe dies
      nun als ein lineares Gleichungssystem, mit
      $x=(x_1,\ldots,x_N),a_i=(a_i^1,\ldots,a_i^N),%
      t=(t_1,\ldots,t_n,0,\ldots,0),B=(b_i^j)_{i,j}$.
      Somit schreibt sich die obige Gleichung wie folgt:
      $x-a_0 = B\cdot t$.
     
      Forme nach $t$ um. Dies ist möglich da $B$ als darstellende
      Matrix von einer Basis invertierbar ist. Nun wird ersichtlich,
      weshalb die $t_i$ für $1 \leq i \leq n$ stetige Funktionen sind.

      \pagebreak

      Mit $t_0 = 1 - \sp{1}{n} t_i$ ist die Behauptung gezeigt.
      % BEMERKUNG: dies zeigt auch die eindeutigkeit der t_i, da
      % lineares gleichungssystem, eindeutige lösung
    \end{description}
  \end{proof}
\end{Lem}

Die Baryzentrischen als ausgezeichnete Koordinaten ermöglichen eine
gute Art der Beschreibung der Punkte innerhalb des Simplex. Mit dieser
Darstellung ist es leicht, möglichst viele Eigenschaften von Simplizialen
zu zeigen.

\begin{Def}[Eckmenge, Dimension, Seite, Rand, Inneres]
  Sei $\sigma = a_0 \ldots a_n$ ein geometrischer $n$-Simplex, so
  definiere folgende geometrische Objekte
  \begin{enumerate}[\textbullet]%{\bfseries1)}]
  \item Die Menge $\{ a_0 , \ldots , a_n \}$ bezeichnet man als
    \textbf{Eckmenge} $\V(\sigma)$ von $\sigma$.
  \item Eine Teilmenge $\tau \subset \sigma$ heißt \textbf{Seite},
    falls $\tau$ einen Simplex bildet. Eine Seite heißt \textbf{echt},
    falls sie von $\sigma$ verschieden ist.
  \item Die \textbf{Dimension} von $\sigma$ ist die Zahl $n$ bzw.
    $\dim(\sigma) = \gr{\V(\sigma)} - 1$, wobei $\dim(\emptyset)=-1$
    gesetzt wird.
  \item Der \textbf{Rand} von $\sigma$ ist die folgende Menge:
    \begin{gather*}
      \partial\sigma \coloneqq \bigcup \; \left\{ \tau \; \Big| \; \tau \text{
          ist echte Seite von } \sigma \right\}.
    \end{gather*}
  \item Das \textbf{Innere} des Simplex ist die Menge
    \begin{gather*}
    	\Int(\sigma) \coloneqq \sigma \setminus \partial\sigma.
    \end{gather*}
  \end{enumerate}
\end{Def}

Zur vorherigen Definition ein anschauliches Beispiel.

\begin{Bsp}
  Betrachte den Simplex $\sigma^2 = e_0e_1e_2$, dann lassen sich alle
  vorherigen Begriffsbildungen leicht anhand dieses Simplex angeben.
  \newline

\centering
\parbox{0.7\linewidth}{% 
\begin{tikzpicture}
    \node (s) at (0,1.5) {$\dim(\sigma^2)=2$};

  \draw [thin,fill=lightgray] (90:1) to (210:1) to (330:1) to (90:1);
    \node (title) at (0,0) {$\sigma^2$};
    \fill (90:1) circle[radius=0.07cm];
    \fill (210:1) circle[radius=0.07cm];
    \fill (330:1) circle[radius=0.07cm];
\end{tikzpicture}
\hfill
\raisebox{0.8cm}{$=$}
\hfill
\begin{tikzpicture}
  \draw [thin] (90:1) to (210:1) to (330:1) to (90:1);
    \node (title) at (0,0) {$\partial\sigma^2$};
    \fill (90:1) circle[radius=0.07cm];
    \fill (210:1) circle[radius=0.07cm];
    \fill (330:1) circle[radius=0.07cm];

\end{tikzpicture}
\hfill
\raisebox{0.8cm}{$\cup$}
\hfill
\begin{tikzpicture}
  \draw [dashed,fill=lightgray] (90:1) to (210:1) to (330:1) to (90:1);
    \node (title) at (0,0) {$\Int\sigma^2$};
\end{tikzpicture}
}
\end{Bsp}
% TODO: tikz bilder einfügen von beispielen in denen für ein beispiel symbolisch die obrigen definitionen angegeben werden

Es werden nun einige Charakterisierungen und Aussagen über die
Simplizes und deren geometrische Objekte bewiesen.

\begin{Satz}\label{satz:simp}
  \normalfont Sei $\sigma = a_0 \ldots a_n $ ein $n$-Simplex und
  $x \in \sigma$ mit der baryzentrischen Darstellung
  $x=\sum\limits_{i=0}^n t_i a_i$, so gelten folgende Aussagen
  \begin{enumerate}[(a)]
  \item
    $x \in \partial\sigma \Leftrightarrow \exists \; 0 \leq i \leq n :
    t_i = 0$ und \label{satz:a}
    $x \in \Int(\sigma) \Leftrightarrow \forall \; 0 \leq i \leq n :
    t_i > 0$
  \item Jeder Simplex $\sigma$ ist eine konvexe, kompakte Teilmenge (
    vom $\R^N$), insbesondere ist die konvexe Hülle von
    $\{ a_0 \ldots a_n \}$ identisch mit dem Simplex $\sigma$.
  \item Das Innere $\Int(\sigma)$ ist offen und konvex.
  \item Zwei Simplexe der selben Dimension sind homöomorph.
%unwichtige aussage
%  \item Es gilt $\overline{\Int(\sigma)} = \sigma$.
  \item Es gibt einen Homöomorphismus $\sigma \simeq \Bn$, der den
    Rand $\partial\sigma$ auf die $\Sp^{n-1}$ abbildet.
  \end{enumerate}
  \begin{proof}
    \begin{enumerate}[(a):]
      % a)
    \item
      \begin{description}
      \item[\glqq $\Rightarrow$\grqq] Sei $x$ im Rand
        $\partial\sigma$, so liegt der Punkt in einer echten Seite
        $\tau$ von $\sigma$. Sei \OE~ $\tau = a_0 \ldots a_m$ für ein
        $m < n$, dann hat $x$ eine eindeutige baryzentrische
        Darstellung bezüglich des Simplex $\sigma$, aber auch eine
        eindeutige Darstellung bezüglich der Seite $\tau$. Es
        existiert also $t_i,s_i$, mit
        \begin{gather*}
          x = \sp{0}{m} s_i a_i = \sp{0}{n} t_i a_i .
        \end{gather*}
        Durch die Eindeutigkeit der Darstellung folgt unmittelbar,
        dass ein $0 \leq i \leq n$ existieren muss, so dass $t_i = 0$
        gilt.
      \item[\glqq $\Leftarrow$ \grqq] Gilt nun umgekehrt $t_i = 0$ für
        ein $0 \leq i \leq n$, so liegt der Punkt $x$ in der echten Seite
        $a_0 \ldots a_{i-1} a_{i+1} \ldots a_n$ und damit im Rand.
      \end{description}

      Die zweite Aussage folgt unmittelbar als Negation von a) und der
      Tatsache, dass $\sigma = \partial\sigma \cup \Int(\sigma)$ eine
      disjunkte Vereinigung ist.

      % b)
    \item 
      \begin{description}
      \item[kompakt:] Definiere eine stetige Abbildung
        $f : \R^{n+1} \rightarrow \R^N$ mit
        $ t_0,\ldots ,t_n \mapsto \sp{0}{n} t_i a_i$, diese ist stetig
        als Linearkombination stetiger Funktionen.  Die Menge
        \begin{gather*}
          A= \set{ (t_0,\ldots,t_n) \in \R^{n+1}}{ \sp{0}{n} t_i = 1
            \text{ und für alle } i , t_i > 0}
        \end{gather*}
        ist kompakt, da man sie durch die $1$-Norm $\nn_1$ wie folgt
        schreiben kann $A = (\nn_1)^{-1}(\{ 1 \}) \cap
        [0,1]^{n+1}$.
        Als Schnitt einer abgeschlossen und einer kompakten Menge ist
        $A$ kompakt.  Somit ist $\sigma = f(A)$ als Bild einer
        kompakten Mengen unter einer stetigen Abbildung wieder
        kompakt.
      \item[konvex:] Seien zwei Punkte $x,y \in \sigma$ gegeben und
        ihre baryzentrischen Darstellungen seien $x = \sum t_i a_i$,
        $y = \sum s_i a_i$ mit $\sum t_i = \sum s_i = 1$. Dann folgt
        für eine Konvexkombination mit $\lambda \in [0,1]$
        \renewcommand*{\theequation}{$*$}
        \begin{align}
          \nonumber
          \lambda x + (1- \lambda)y &= \lambda \cdot \sp{0}{n} t_i a_i%
                                      + (1-\lambda) \cdot \sp{0}{n} s_i a_i \\
                                    &= \sp{0}{n} (\lambda t_i +%
                                      (1-\lambda) s_i) \cdot a_i  
        \end{align}
        Jeder Punkt dieses Verbindungsstücks von $x$ und $y$ liegt
        wieder in $\sigma$, den mit den Koeffizienten aus der
        Darstellung $(*)$ folgt
        \begin{align*}
          \sp{0}{n} \lambda t_i + (1-\lambda) s_i 
          &= \lambda \cdot \sp{0}{n} t_i + (1-\lambda) \cdot \sp{0}{n} s_i \\
          &= \lambda + (1 - \lambda) = 1.
        \end{align*}
        Also folgt $\lambda x + (1- \lambda)y \in \sigma$.  

      \item[Konvexe Hülle:] Es ist zu zeigen
        $\conv(\sset{a_0,\ldots , a_n}) = \sigma$. Hierbei gilt
        $\sset{a_0,\ldots,a_n} \subset \sigma$. Da $\conv$ monoton
        ist, folgt $\conv(\sset{a_0,\ldots , a_n}) \subset \sigma$.
        
        Nun gilt $a_ia_j \subset \conv( \sset{a_0,\ldots , a_n} )$, da
        mit $a_i \in \ch$ auch jeder Punkt, der sich als
        Konvexkombinationen schreiben lässt in der konvexen Hülle
        liegt. Rekursiv folgt nun mit der Darstellung
        $x = t_0 a_0 + \lambda \cdot \sp{1}{n} \frac{t_i}{\lambda}
        a_0$ die andere Inklusion.
     \end{description}

     % c)
   \item \begin{description}
     \item[offen:] Da jede Seite $\tau \subset \sigma$ wieder ein
       Simplex ist, ist diese auch kompakt, insbesondere
       abgeschlossen. Nun ist der Rand $\partial\sigma$ Vereinigung
       dieser abgeschlossen Mengen, also wieder abgeschlossen. Das
       Innere ist nun das relative Komplement der abgeschlossen Menge
       bezüglich des Simplex, also offen.
     \item[konvex:] Seien zwei Punkte $x,y \in \Int(\sigma)$ gegeben,
       dann gilt nach $b)$, dass für die baryzentrischen Koordinaten
       der Punkte~~$t_i,s_i >0$ gilt. Sei nun $\lambda \in [0,1]$, so
       gilt für die baryzentrischen Koordinaten der
       Konvexkombinationen $\lambda x + (1-\lambda y)$ die Darstellung
       $\lambda t_i + (1-\lambda)s_i$. Dies ist offentsichlich stets
       größer als $0$.
        \end{description}

        % d)
      \item Zeige, dass ein beliebiger Simplex $\sigma^n$ homöomorph zum
        Standardsimplex $e_0\ldots e_n$ ist. Betrachte hierfür die
        affine Transformation $f(x) = Ax+b$ mit $A \in \GL(N,\R)$ und
        $b \in \R^N$, die $a_i$ auf $e_i$ abbildet. Dies ist ein
        Homöomorphismus.
      % \item Da $\sigma$ abgeschlossen ist und
      %   $\overline{\hspace{0.1cm} \cdot \hspace{0.1cm}}$ ist monoton
      %   gilt: $\overline{\Int(\sigma)} \subset \sigma$.

      %   Sei nun $x \in \sigma$. Nutze die disjunkte Zerlegung des
      %   Simplex $\sigma = \partial\sigma \cup \Int(\sigma)$. Dann
      %   unterscheide die beiden Fälle das $x$ in genau einem der
      %   beiden Mengen liegt. Für
      %   $x \in \Int(\sigma) \subset \overline{\Int(\sigma)}$ sind wir
      %   schon fertig. Für $x \in \partial\sigma$ nutze die
      %   Charakterisierung der Elemente des Randes, also
      %   $x \in \partial A \Leftrightarrow \forall U \text{ offene
      %     Umgebung von } x : U \cap A \not= \emptyset \text{ und } U
      %   \cap A^c \not= \emptyset $.
      %   Und mithilfe der Charakterisierung der Elemente des
      %   Abschlusses:
      %   $ x \in \overline{A} \Leftrightarrow \forall U \text{ offene
      %     Umgebung von }x : U \cap A \not= \emptyset$.
      %   Und somit folgt die Behauptung.

      %   e)
      \item Betrachte die stetige Abbildung
        $f : \R^{n+1} \setminus \sset{0} \rightarrow \Sp^{n}, f(x) =
        \frac{x}{\nn[x]}$.
        Definiere für
        $a \in \Int(\sigma), p \in \R^N \setminus \{ 0 \}$ die
        Halbgeraden $ap_+ \coloneqq \set{ a + tp}{ t \geq 0}$. Dann
        beweise folgende Behauptung.
      \begin{Beh}
        Der Schnitt $\partial\sigma \cap ap_+$ hat genau ein Element.
        \begin{proof}
          Zur Existenz des Element betrachte die Menge
          $ap_+ \cap \Int(\sigma)$.  Diese ist homöomorph zu $[0,b)$
          für ein $b>0$ und somit existiert bezüglich des
          Homöomorphismus durch den Punkt $b$ ein Schnittpunkt der
          Halbgerade mit dem Rand.
          
          Seien nun zwei Punkte $x,y \in \partial\sigma$, die auch in
          $ap_+$ enthalten sind. Seien $t_x, t_y \in \R_{> 0}$ die
          Parameter, sodass $a + t_i \cdot p = i$ mit
          $i \in \sset{x,y}$ gilt. Sei \OE~ $t_y < t_x$, dann folgt
          durch eleminieren von $p$
          \begin{gather*}
            x = (1-t) a + ty \text{ mit } t=\frac{t_y}{t_x} < 1.
          \end{gather*}
          Wähle nun eine Folge $y_n \in \Int(\sigma)$ die gegen $y$
          konvergiert und definiere
          $a_n \coloneqq \frac{x -ty_n}{1-t}$.  Dann konvergiert nach
          Konstruktion $a_n$ gegen $a$ und da $a \in \Int(\sigma)$ in
          einer offenen Menge enthalten ist, existiert ein $n \geq 0$,
          sodass für alle $m \geq n$ gilt $a_m \in
          \Int(\sigma)$.
          Somit gilt $x = (1-t) a_n + ty_n \in \Int(\sigma)$, da die
          Menge konvex ist und die Folgenglieder ab einem $m$ in der
          Menge $\Int(\sigma)$ liegen, folgt ein Widerspruch zu
          $x \in \partial\sigma$.
        \end{proof}
      \end{Beh}
      Sei nun \OE~ $0 \in \Int(\sigma)$, sonst wende eine affine
      Transformation auf $\sigma$ an, die den Nullpunkt in das Innere
      von $\sigma$ verschiebt.  
%TODO: genauer erklären
      Schränke nun die Abbildung $f$ auf die
      Menge $\partial\sigma$ ein. Nach obiger Behauptung ist die
      Abbildung
      $f_{| \partial\sigma} : \partial\sigma \rightarrow \Sp^{n-1}$
      bijektiv und stetig. Da $\partial\sigma$ und $\Sp^{n-1}$ kompakt und
      hausdorffsch sind, ist die Abbildung auch ein Homöomorphismus.
      
      Setze nun als Umkehrabbildung $g : \Sp^{n-1} \rightarrow \partial\sigma$ 
      und erweitere diese zu einem Homöomorphismus der Form
      \begin{gather*}
        G : \Bn \rightarrow \sigma , \hspace{0.5cm}
        x \mapsto 
        \begin{cases}
          \hspace{.7cm}0 & , x = 0. \\
          ~\nn[g(\frac{x}{\nn[x]})] \cdot x & \text{, sonst}.
        \end{cases}
      \end{gather*}
      Diese Abbildung erfüllt die gewünschten Eigenschaften.
    \end{enumerate}
  \end{proof}
\end{Satz}


\begin{Bem}
  Für eine beliebige Indexmenge $J$ ist der Raum der Funktionen $\R^J$ ein
  $\R$-Vektorraum, dessen Elemente als Tupel $(x_j)_{j \in J}$
  geschrieben werden. Betrachte den Untervektorraum
  $\E^J \coloneqq \bigoplus\limits_{j \in J} \R$ der Elemente, die bis
  auf endlich viele von Null verschieden sind. Definiere für zwei
  Elemente $x,y \in \E^J$ eine Metrik
  \begin{gather*}
    \gr{x-y} \coloneqq \sup \set{ \gr{x_j - y_j } }{ j \in J}.
  \end{gather*}
  Die obigen Definitionen und Aussagen funktionieren ebenso falls man
  die Simpliziale in dem Raum $\E^J$ betrachtet. Somit ist sind auch
  geometrische Simpliziale der Kardinalität größer als der von $\R$
  möglich.
\end{Bem}




%%% Local Variables:
%%% mode: latex
%%% TeX-master: "main"
%%% End:


%% kapitel über geometrische simplizialkomplexe und simplizialabbildunge
% geometrische simplizialkomplexe

\section{Geometrische Simplizalkomplexe}

%TODO: bessere zusammenfassung schreiben
Die geometrischen Simpliziale allein reichen nicht aus um beliebige 
Teilmenge des $\R^N$ zu beschreiben, es ist noch notwendig eine Ansammlung
von Simpliziale auf richtige art und Weise beschreiben zu können. Hierzu
sind die Simplizialkomplexe da.

\begin{Def}[Geometrischer Simplizialkomplex]
	Wir nennen eine Menge $\Delta$ von (geometrischen) Simplizialen einen \textbf{geometrischen Simplizialkomplex}
	falls er folgende Bedingungen erfüllt:
	\begin{enumerate}[(K1)]
		\item $\emptyset \in \Delta$
		\item Für jeden Simplex $\sigma \in \Delta$ ist auch jede Seite
		$\tau$ in $\Delta$ enthalten
		\item Mit allen Simplexen $\sigma, \sigma' \in \Delta$ ist auch ihr
		Schnitt $\sigma \cap \sigma'$ in $\Delta$ enhalten.
	\end{enumerate}
\end{Def}

\begin{Def}[Teilkomplex, $k$-Skelett, Dimension]
	Sei $\Delta$ ein Simplizialkomplex, so definiere
	\begin{enumerate}[1)]
		\item Eine Teilmenge $\Delta'$ von $\Delta$ heißt Teilkomplex, falls
		$\Delta'$ diese Menge wiederum einen Simplizialkomplex bildet.
		\item Die Dimension eines Simplizalkomplexes ist wie folgt
		definiert:
		\begin{gather*}
			\dim(\Delta) \coloneqq \sup \left\{ \dim(\sigma) \; \Big| \;
			\sigma \in \Delta \right\}
		\end{gather*}
		die Dimension kann auch unendlich sein.
		\item Das $k$-Skelett $\Delta^k$ ist der Teilkomplex aller Simplizes
		mit der Dimension kleiner gleich $k$.  Ein Spezialfall davon ist
		die Eckmenge, dies entspr
	\end{enumerate}
\end{Def}

%TODO: zeige das dies topologie definiert
\begin{Def}[schwache Topologie]
	Sei $\Delta$ ein Simplizalkomplex, dann definiere die
	\textbf{schwache Topologie} auf der geometrischen Realisierung $| \Delta |$,
	durch folgende Charakterisierung der abgeschlossenen Mengen.
	\begin{gather*}
		A \subset \Delta \text{ abgeschlossen } :\Leftrightarrow A \cap
		\sigma \text{ abgeschlossen für alle } \sigma \in \Delta
	\end{gather*}
	Wobei auf der rechten Seite die Menge abgeschlossen bezüglich der
	Standardtopologie auf dem $\R^{\dim(\Delta)}$ ist.
\end{Def}

\begin{Lem}
	Die schwache Topologie ist feiner als die Standardtopologie und
	falls $\Delta$ endlich ist stimmen Standardtopologie, schwache
	Topologie auf $\gr{\Delta}$ überein.
	\begin{proof}
		Sei \OE\; $\gr{\Delta} \subset \R^N$ und $A \subset \gr{\Delta}$
		eine abgeschlossene Menge, so existiert ein $B \subset \R^N$
		abgeschlossen so dass $A = B \cap \gr{\Delta}$ gilt. Nun gilt für
		jeden Simplex $\sigma \in \Delta$ folgende Gleichheit:
		$A \cap \sigma = B \cap \gr{\Delta} \cap \sigma = B \cap \sigma$
		und somit ist $A \cap \sigma$ abgeschlossen bezüglich $\sigma$,
		also abgeschlossen bezüglich der schwachen Topologie.
		
		Falls nun $\Delta$ eine endliche Menge von Simplizes ist, gilt für
		eine abgeschlossene Menge $A$ bezüglich $\gr{\Delta}$ das
		$A \cap \sigma$ abgeschlossen in $\sigma$ ist und durch
		Vereinigung über die endlichen vielen Simplizes die
		Abgeschlossenheit bezüglich der Standardtopologie erreicht wird.
	\end{proof}
\end{Lem}

\begin{Bem}
	Die schwache Topologie ist im allgemeinen feiner als die
	Spurtopologie auf $\gr{\Delta}$ bezüglich dem $\R^N$. Betrachte
	hierzu den folgenden Simplizialkomplex,
	$\Delta \coloneqq \set{ \{t\} }{t \in \R}$.  Hierbei ist die
	Geometrische Realisierung gleich zu $\R$, aber die schwache
	Topologie entspricht der Diskreten, denn für eine beliebige
	Teilmenge von $A \subset \R$ ist der Schnitt $A \cap \{ t \}$ für
	ein beliebigen $0$-Simplex aus $\Delta$ gleich dem Simplex selbst,
	somit stets abgeschlossen, also ist jede Teilmenge abgeschlossen und
	damit die schwache Topologie auf der Realisierung der diskreten und
	damit ungleich der Standardtopologie auf $\R$.
\end{Bem}

%TODO: topologie von sim komplexen ist hausdorffsch

%TODO: ziegler - literatur : seite 11 (in reader abgelesen)



% \begin{Def}[Topologie auf dem Träger eines Simplizialkomplexes]
%   Eine Menge $U \subset \left| \Delta \right|$ ist genau dann offen,
%   falls für jeden Simplex $\sigma \in \Delta$ die Menge
%   $U \cap \sigma$ offen im unterliegenden Raum $\R^n$ ist.
% %	also bezüglich der spurtopologie auf \sigma
% % sie elementsOfalgtop seite 111 , review of quotient spaces, für andere beschreibung der topologie
% \end{Def}


%definitionen: unterkomplex, geometrische realisierung, k-skelett, schwache topologie, simpliziale abbildung, simplizialer iso, 

%aussagen: 	-	für endliche komplexe stimmen topologien überein
%			-	|\Delta| ist hausdorffsch
%			- 	Aussagen über kompaktheit, und kompakte teilmengen von \Delta
%			-	



\begin{Def}[Geometrische Realisierung]
	Die Geometrische Realisierung eines Simplizalkomplexes $\Delta$ ist
	die Vereinigung all seiner Elemente.
	\begin{gather*}
	|\Delta| \coloneqq \bigcup \left\{ \sigma \; \Big| \; \sigma \in
	\Delta \right\} \subset \R^{\dim(\Delta)}
	\end{gather*}
\end{Def}








%%% Local Variables:
%%% mode: latex
%%% TeX-master: "main"
%%% End:

%% kapitel über abstrakte simplizialkomplexe
%% abstrakte simplizialkomplexe

\section{Abstrakte Simplizialkomplexe}

Für topologische Anwendung ist der $\R^J$ in vielen Fällen zu konkret
und unnötiger gedanklicher Balast. Deshalb sollte man die
Simplizialkomplexe unabhängig vom euklidischen Raum definieren. Diese
abstrakten Simplizialkomplexe werden in diesem Abschnitt
behandelt. Die zunächst stark erscheinende Vereinfachung, stellt sich
als eine äquivalente Darstellung zu den geometrischen Komplexen
heraus.

\begin{Def}
  Eine beliebige, nichtleere Menge $\Delta$ heißt \textbf{abstrakter
    Simplizialkomplex} oder auch nur abstrakter Komplex, falls mit
  jedem Element aus $\Delta$ auch jede Teilmenge aus diesem Element in
  $\Delta$ enthalten ist und jedes Element eine endliche
  Menge ist. Oder in Formeln geschrieben
%  \setcounter{equation}{(*)}
\renewcommand*{\theequation}{\textbullet}
  \begin{gather}
    \sigma \in \Delta \text{ und }  \tau \subset \sigma
    \Rightarrow \tau \in \Delta
  \end{gather}
  Für einen abstrakten Komplex werden folgende Begriffe benötigt
  \begin{enumerate}[({A}1)]
% dimension, seite, knoten, unterkomplex, simpliziale abbildung, isomorphie
  \item Ein Element des Komplexes heißt \textbf{Seite}. Die Teilmenge aller
    Seiten der Mächtigkeit kleiner gleich $k$ heißt das \textbf{$k$-Skelett},
    insbesondere für $k=1$ ist dies die \textbf{Knotenmenge}.
  \item Die \textbf{Dimension} des Komplexes ist das Supremum über die
    Mächtigkeit der Seiten von $\Delta$.
  \item Eine Teilmenge vom Komplex die (\textbullet) erfüllt, heißt
    \textbf{Unterkomplex}.
  \item Eine Abbildung zwischen den Knotenmenge zweier abstrakten
    Komplexe heißt \textbf{simpliziale Abbildung}, falls Seiten auf
    Seiten abgebildet werden.
  \item Zwei abstrakte Komplexe heißen \textbf{isomorph}, falls es eine
    bijektive simpliziale Abbildung gibt, deren Inverses auch
    simplizial ist.
  \item Sei $\Delta$ ein geometrischer Komplex so bezeichnet die Menge
    \begin{gather*}
      \set{ \sset{a_0 \ldots a_n} \subset \Delta^{(0)} }{ a_0 \ldots
        a_n \in \Delta}
    \end{gather*}
    das \textbf{Knotenschema} von $\Delta$.
  \item Ist ein abstrakter Komplex $\Delta$ isomorph zum Knotenschema
    eine geometrischen Komplexes $\Delta'$, so nennen wir $\Delta'$
    eine \textbf{geometrische Realisierung} von $\Delta$.
  \end{enumerate}
\end{Def}

\begin{Bsp}
  \begin{enumerate}[\textbullet]
  \item Für einen geometrischen Komplex $\sigma$ ist die Menge all
    seiner Seiten ein abstrakter Simplizalkomplex und das Knotenschema
    wird auf triviale Art und Weise geometrisch durch $\sigma$
    realisiert.
  \item Die Menge $\Delta = \sset{%
      \sset{a,b},\sset{a,c},%
      \sset{b,c},\sset{a},\sset{b},\sset{c},\emptyset}$ ist ein Beispiel und 
    seine geometrische Realisierung sieht wie folgt aus
    \begin{center}
      \begin{tikzpicture}
        \draw [thin] (90:1) to (210:1) to (330:1) to (90:1);
        \foreach \foo in {90,210,330}
        \fill (\foo:1) circle[radius=0.07cm];
      \end{tikzpicture}
    \end{center}

  \item Betrachte die Menge
    $\set{ \sset{i,i+1} , \sset{i} }{ i \in \Z} \cup \sset{\emptyset}$
    dann ist der Polyeder zu einer geometrischen Realisierung
    homöomorph zu $\R$

  \item Für eine beliebige Menge ist die Potenzmenge ohne die leere
    Menge stets ein abstrakter Simplizialkomplex
  \item Die von-Neumann Zahlen liefern einen nicht endlichen
    abstrakten Komplex.  Diese Menge erhält man wie folgt, setze
    $V_0 \coloneqq \emptyset$ und
    $V_{n+1} \coloneqq \sset{V_n} \cup \V_n$, dann ist der Komplex
    $V \coloneqq \bigcup\limits_{n \geq 0} V_n$ nicht endlich.
  \item Ein ungerichteter Graph $G=(V,K)$ mit $V$ die Eckmenge und $K$
    die Kanten ist ein abstrakter Simplizialkomplex
  \item Ein Knotenschema von einem geometrisch Komplex ist ein
    abstrakter Komplex (siehe unten)
  \end{enumerate}
\end{Bsp}

% TODO: zeige das für ein element aus \bigcup K ein eindeutiges
% Element aus K existiert so dass das element aus \Int dieses element
% ist

Der Zusammenhang zwischen geometrischen und abstrakten Komplexen ergibt sich 
aus der folgenden Definition.

\begin{Def}[Knotenschema]
\end{Def}

\begin{Bsp}
  foo
  % TODO: füge ein bild von einen geo komplex ein, aus dem das
  % knotenschema extrahiert wird
\end{Bsp}

\begin{Satz}
  Es gelten folgende Aussagen
  \begin{enumerate}[(1)]
  \item Für jeden abstrakten Komplex existiert ein geometrischer
    Komplex dessen Knotenschema (abstrakt) isomorph zu diesem Komplex
    ist.
  \item Zwei geometrisch Komplexe sind (linear) isomorph, genau dann
    wenn die entsprechenden Knotenschemata (abstrakt) isomorph sind.
  \end{enumerate}
  \begin{proof}
%TODO: noch zu beweisen
    foo
  \end{proof}
\end{Satz}

% todo: topologie --literatur/archiv seite 333ff

% erkläre das labeln von triangulierungen, bzw
% geometrischen,abstrakten simplizialkomplexen

\begin{Bem}[Label,Abwicklung]
  Da im weiteren Verlauf des Seminars meist nicht zweidimensionale
  Triangulierungen angegeben und benötigt werden, ist es notwendig
  eine vereinfachte Darstellung dieser höher dimensionalen Objekte
  abzugeben. Hierzu wird das Labeln von Triangulierungen angegeben.

  Betrachte zunächst für die Idee des Verfahrens den Tetraeder und
  seine standard Abwicklung im Zweidimensionalen.
  % zeichne dreidim tetraeder, und eine zweidim abwicklung, tikz
  Bei der Abwicklung werden die $0$-Simplexe fortlaufend alphabetisch
  nummeriert. Bei Knoten die in der ursprünglichen Dimension
  zusammenfallen wird das gleiche Label verwendet.

%TODO: auch für unendlich dimensionale abwicklung
  Die Formalisierung der obigen Idee wird durch die Angabe eines
  abstrakten Komplexes gegeben, dessen geometrische Realisierung dem
  geometrischen Komplex entspricht. Sei $\Delta$ ein endlich
  dimensionaler geometrischer Komplex gegeben. Dann ist die Abwicklung
  durch einen zweidimensionalen geometrischen Komplex $\Delta'$ und
  eine surjektive Abbildung $f: \Delta \rightarrow \Delta'$ gegeben.

%TODO: weiter ausführen

\end{Bem}

% TODO: label und zweidim abwicklungen vom torus, zylinder, kugel und kleinsche flasche

%einfache beispiele

\begin{Bsp}
  Es werden nun Triangulierungen von für Kugel, Zylinder, Torus angegeben.
  Wobei auch die nicht Eindeutigkeit von Triangulierungen anhand des Torus
  aufgezeigt werden.
\end{Bsp}

\begin{Bsp}[Triangulation]
%TODO: beweise folgende triangulierengen \S^2 \simeq \gr{(\Delta^3)^{(2)}}, behauptung diese aussage gilt für allgemeines n, also \S^n \simeq \gr{(\Delta^{n+1})^{(n)}}
\end{Bsp}





% abzählbare mengen können nur durch 0-simplexe trianguliert werden



%Begriffe:
%geometrisch unabhängig, und linear unabhängig vergleichen gegenüberstellen
% simplex, geometrischer simplizialkomplex, abstrakter simplizialkomplex 


% n-ebene, menge von punkten, aufgespannt durch geo.unab system mit konvexkombinationen
% die faktoren sind eindeutig bestimmt

%affine transformation: x -> Ax+b

%geo.unab systeme werden durch affine transformationen auf geo.unab systeme abgebildet

% n simplex

% baryzentrische koordinaten

% ist die leere menge eine simplizial?

% ein simplex ist genau die konvexe hülle von einer endlichen teilmenge vom \R^N

% eine seite eines simplex ist ein teilsimplex der dimension d

% anzahl der d teilsimplexe ist binomial_koeff(n+1,d+1), für einen n dim simplex
% klar wähle aus den n+1 punkten d+1 stück aus

% verwende das schläfli symbol zur beschreibung der umgebung eines
% punktes aus einem simplex für triangulierungen werden nur 2fache
% schläflisymbole benötigt, also {p,q}, da bei der triangulierung nur
% \laplace^2 simplexe verwendet werden, vereinfacht sich das symbol
% auf die form {3,q}. gebe nun zu jedem punkt auf der
% mannigfaltigkeit, also den 0 dimensionalen simplexen, die anzahl der
% angrenzenden dreiecker/laplace^2 simplexe an


% schreibe in die appendix ein eigenen anhang nur mit getikzten beispielen
% die komplette pflasterung von \R^2 oder allgemein die füllung des \R^n mit \laplace^n
% verschiedene triangulierungen

% zeige nicht-kompakte teilmengen vom \R^n sind nicht triangulierbar,
% einfachstes bsp (0,1), einfach das geometrische realisierungen stets
% kompakte mengen sind, somit kann es keinen homöomorphismus geben


%%% Local Variables:
%%% mode: latex
%%% TeX-master: "main"
%%% End:
%\printindex

%\index{marker}
%\nocite{*}

%TODO: Literatur


%TODO: stichwortverzeichnis mit makeindex setzen
%\printbibliography

\end{document}
